%2multibyte Version: 5.50.0.2953 CodePage: 65001
\documentclass{amsart}%
\usepackage{amsmath}
\usepackage{amssymb}
\usepackage{amsfonts}
\usepackage{geometry}
\usepackage{color}
\usepackage{amsmath}
\usepackage{verbatim}
%\usepackage[draft]{graphicx}%
\usepackage{graphicx}%

\usepackage[outer]{showlabels}

\setcounter{MaxMatrixCols}{30}
%TCIDATA{OutputFilter=latex2.dll}
%TCIDATA{Version=5.50.0.2953}
%TCIDATA{Codepage=65001}
%TCIDATA{CSTFile=amsartci.cst}
%TCIDATA{Created=Wednesday, May 12, 2021 11:27:07}
%TCIDATA{LastRevised=Wednesday, September 06, 2023 15:11:48}
%TCIDATA{<META NAME="GraphicsSave" CONTENT="32">}
%TCIDATA{<META NAME="SaveForMode" CONTENT="1">}
%TCIDATA{BibliographyScheme=Manual}
%TCIDATA{<META NAME="DocumentShell" CONTENT="Articles\SW\AMS Journal Article">}
%BeginMSIPreambleData
\providecommand{\U}[1]{\protect\rule{.1in}{.1in}}
\renewcommand{\hat}{\widehat}
\renewcommand{\tilde}{\widetilde}
\renewcommand{\epsilon}{\varepsilon}
\newcommand{\stopp}{\mathcal{S}}
\newcommand{\roof}{\mathcal{R}}
\newcommand{\supr}{\mathcal{SU}}
%\renewcomamnd{\bar}{\overline}
%EndMSIPreambleData
\newtheorem{theorem}{Theorem}
\theoremstyle{plain}
\newtheorem{acknowledgement}[theorem]{Acknowledgement}
\newtheorem{algorithm}[theorem]{Algorithm}
\newtheorem{axiom}[theorem]{Axiom}
\newtheorem{case}[theorem]{Case}
\newtheorem{claim}[theorem]{Claim}
\newtheorem{conclusion}[theorem]{Conclusion}
\newtheorem{condition}[theorem]{Condition}
\newtheorem{conjecture}[theorem]{Conjecture}
\newtheorem{corollary}[theorem]{Corollary}
\newtheorem{criterion}[theorem]{Criterion}
\newtheorem{definition}[theorem]{Definition}
\newtheorem{example}[theorem]{Example}
\newtheorem{exercise}[theorem]{Exercise}
\newtheorem{lemma}[theorem]{Lemma}
\newtheorem{notation}[theorem]{Notation}
\newtheorem{problem}[theorem]{Problem}
\newtheorem{proposition}[theorem]{Proposition}
\newtheorem{remark}[theorem]{Remark}
\newtheorem{solution}[theorem]{Solution}
\newtheorem{summary}[theorem]{Summary}
\numberwithin{equation}{section}
\geometry{left=1in,right=1in,top=1in,bottom=1in}
\begin{document}
\title{Haar basis and frame testing}
\author[M. Alexis]{Michel Alexis}
\address{Mathematical Institute, University of Bonn, Endenicher Allee 60, 53115, Bonn, Germany}
\email{micalexis.math@gmail.com}
\author[J. L. Luna-Garcia]{Jose Luis Luna-Garcia}
\address{Department of Mathematics \& Statistics, McMaster University, 1280 Main Street
West, Hamilton, Ontario, Canada L8S 4K1}
\email{lunagaj@mcmaster.ca}
\author[E.T. Sawyer]{Eric T. Sawyer}
\address{Department of Mathematics \& Statistics, McMaster University, 1280 Main Street
West, Hamilton, Ontario, Canada L8S 4K1 }
\email{Sawyer@mcmaster.ca}
\thanks{M. Alexis is funded by the Deutsche Forschungsgemeinschaft (DFG, German
Research Foundation) under Germany's Excellence Strategy - GZ 2047/1,
Projekt-ID 390685813.}
\thanks{E. Sawyer is partially supported by a grant from the National Research Council
of Canada}

\begin{abstract}
We show that for two doubling measures $\sigma$ and $\omega$ on $\mathbb{R}%
^{n}$, any gradient-elliptic smooth $\lambda$-fractional Calder\'{o}n-Zygmund
operator $T^{\lambda}$, and any \emph{fixed} dyadic grid $\mathcal{D}$ in
$\mathbb{R}^{n}$,
\[
\mathfrak{N}_{T^{\lambda}}\left(  \sigma,\omega\right)  \approx\mathfrak{H}%
_{T^{\lambda}}^{\mathcal{D},\operatorname*{glob}}\left(  \sigma,\omega\right)
+\mathfrak{H}_{T^{\lambda}}^{\mathcal{D},\operatorname*{glob}}\left(
\sigma,\omega\right)  \ ,
\]
where $\mathfrak{N}_{T^{\lambda}}\left(  \sigma,\omega\right)  $ denotes the
operator norm of $T_{\sigma}^{\lambda}$ from $L^{2}\left(  \sigma\right)  $ to
$L^{2}\left(  \omega\right)  $, and%
\[
\mathfrak{H}_{T^{\lambda}}^{\mathcal{D},\operatorname*{glob}}\left(
\sigma,\omega\right)  \equiv\sup_{I\in\mathcal{D}}\left\Vert T_{\sigma
}^{\lambda}h_{I}^{\sigma}\right\Vert _{L^{2}\left(  \omega\right)  }\ ,
\]
is the global Haar testing characteristic for $T^{\lambda}$ on the grid
$\mathcal{D}$, and $\left\{  h_{I}^{\sigma}\right\}  _{I\in\mathcal{D}}$ is
the weighted Haar orthonormal basis of $L^{2}\left(  \sigma\right)  $ arising in the work
of Nazarov, Treil and Volberg.

We then extend this theorem to $L^{p}\left(  \mu\right)$ spaces for
	$1<p<\infty$, using frames in place of orthonormal bases since $\left\{  h_{I}^{\mu}\right\}  _{I\in\mathcal{D}}$ now forms a frame for $L^p \left (\mu \right )$. Namely, we show the norm inequality for $T_{\sigma} ^{\lambda}$ is characterized by global \emph{quadratic} $\mathfrak{H}_{T^{\lambda},p}^{\ell
^{2};\mathcal{D},\operatorname*{glob}}\left(  \sigma,\omega\right)$ Haar
testing characteristics, which are modelled on the vector-valued characteristics
arising in the work of Hyt\'{o}nen and Vuorinen in $L^{p}\left(  \mu\right)$ spaces.

	We also show these theorems extend more generally to weighted Alperted wavelets which replace the weighted Haar wavelets in the proofs of some recent two-weight $T1$ theorems \cite{RaSaWi, AlSaUr, SaWi}.

Finally, we briefly pose these questions in the context of orthonormal bases
in arbitrary Hilbert spaces, and in the context of frames in arbitrary Banach spaces.

\end{abstract}
\maketitle
\tableofcontents


\section{Introduction: orthonormal bases and frames}

Suppose $\mathcal{H}_{1}$ and $\mathcal{H}_{2}$ are separable Hilbert spaces
and that $\mathcal{B}\subset B\left(  \mathcal{H}_{1},\mathcal{H}_{2}\right)
$ is a collection of bounded linear operators from $\mathcal{H}_{1}$ to
$\mathcal{H}_{2}$. A natural question is whether the
collection $\mathcal{B}$ is uniformly bounded in $B\left(  \mathcal{H}%
_{1},\mathcal{H}_{2}\right)  $. The \emph{uniform boundedness principle} says
that this is the case \emph{if and only if} the collection is pointwise
bounded, i.e.%
\[
\sup_{T\in\mathcal{B}}\left\Vert Tf\right\Vert _{\mathcal{H}_{2}}<\infty\text{
for every }f\in\mathcal{H}_{1}\ .
\]
This pointwise criterion requires testing over \textbf{all} $f\in
\mathcal{H}_{1}$, or at least over a subset of second category in
$\mathcal{H}_{1}$, and the question of testing over a smaller set of
functions becomes relevant. A minimal set of\ reasonable testing functions is
given by any fixed orthonormal basis $\mathcal{O}_{1}$ of $\mathcal{H}_{1}$,
and since boundedness of $T$ is equivalent to boundedness of the adjoint
$T^{\ast}$, it is reasonable to include testing the adjoint $T^{\ast}$ over
a\ fixed orthonormal basis $\mathcal{O}_{2}$ of $\mathcal{H}_{2}$. This leads
to the following loosely formulated question for a set $\mathcal{B}$ of
bounded linear operators from $\mathcal{H}_{1}$ to $\mathcal{H}_{2}$ and a
pair of orthonormal bases $\mathcal{O}_{1}$ and $\mathcal{O}_{2}$ of
$\mathcal{H}_{1}$ and $\mathcal{H}_{2}$, respectively.
\begin{align}
  \text{Is }\mathcal{B} &\text{ uniformly bounded in }B\left(
\mathcal{H}_{1},\mathcal{H}_{2}\right)  \text{ if }\label{question} T \text{ and } T^{\ast} \text{ are uniformly bounded on } \\&\mathcal{O}_{1}\text{ and } \mathcal{O}_{2}\text{, respectively, for all }T\in\mathcal{B}%
?\nonumber
\end{align}

To be more precise, suppose $\mathcal{H}_{1}$ and $\mathcal{H}_{2}$ are separable Hilbert spaces
with orthonormal bases $\mathcal{O}_{1}=\left(  e_{k}^{1}\right)
_{k\in\mathbb{N}}$ and $\mathcal{O}_{2}=\left(  e_{k}^{2}\right)
_{k\in\mathbb{N}}$ of $\mathcal{H}_{1}$ and $\mathcal{H}_{2}$ respectively.
Define the norm
\[
\left\Vert T\right\Vert _{\mathcal{H}_{1}\rightarrow\mathcal{H}_{2}}\equiv
\sup_{\left\Vert x\right\Vert _{\mathcal{H}_{1}}=1}\left\Vert Tx\right\Vert
_{\mathcal{H}_{2}}
\]
and the testing characteristics
\[
	\left\Vert T\right\Vert _{\mathcal{O}_{1}}%
\equiv\sup_{k\in\mathbb{N}}\left\Vert Te_{k}^{1}\right\Vert _{\mathcal{H}_{2}%
}\text{ and }\left\Vert T^{\ast}\right\Vert _{\mathcal{O}_{2}}\equiv\sup
_{k\in\mathbb{N}}\left\Vert T^{\ast}e_{k}^{2}\right\Vert _{\mathcal{H}_{1}%
}\ ,
\]
where the name ``testing
characteristic'' arises because the supremum involves testing the action of an 
operator on a collection of ``test functions.'' We are interested in classes of bounded\ linear operators $\mathcal{B}%
\subset B\left(  \mathcal{H}_{1},\mathcal{H}_{2}\right)  $ from $\mathcal{H}%
_{1}$ to $\mathcal{H}_{2}$ whose norms are controlled by their testing conditions on $\mathcal{O}_1$ and $\mathcal{O}_2$, i.e., 
\begin{equation}
\left\Vert T\right\Vert _{\mathcal{H}_{1}\rightarrow\mathcal{H}_{2}}\leq
C\left(  \left\Vert T\right\Vert _{\mathcal{O}_{1}}+\left\Vert T^{\ast
}\right\Vert _{\mathcal{O}_{2}}\right)  ,\ \ \ \ \ \text{for every }%
T\in\mathcal{B}. \label{controlled}%
\end{equation}


It is well known that the answer to Question (\ref{question}) can be negative for
$\mathcal{B}=B\left(  \mathcal{H}_{1},\mathcal{H}_{2}\right)  $ and
arbitrarily prescribed orthonormal bases in separable Hilbert
spaces\footnote{The answer to question (\ref{question}) is trivially
affirmative if we are allowed to choose the orthonormal basis $\mathcal{O}%
_{1}$ depending on $\mathcal{B}$. Indeed, simply choose $T\in\mathcal{B}$ and
$x\in\mathcal{H}_{1}$ such that $\frac{\left\Vert Tx\right\Vert _{\mathcal{H}%
_{2}}}{\left\Vert x\right\Vert _{\mathcal{H}_{1}}}>\frac{1}{2}\sup
_{B\in\mathcal{B}}\sup_{0\neq y\in\mathcal{H}_{1}}\frac{\left\Vert
By\right\Vert _{\mathcal{H}_{2}}}{\left\Vert y\right\Vert _{\mathcal{H}_{1}}}$
(if this supremum is infinite, simply choose $\frac{\left\Vert Tx\right\Vert
_{\mathcal{H}_{2}}}{\left\Vert x\right\Vert _{\mathcal{H}_{1}}}$ arbitrarily
large), and then extend $\left\{  x\right\}  $ to an orthonormal basis
$\mathcal{O}_{1}$.}. Indeed, in the\ single space case $\mathcal{H}%
_{1}=\mathcal{H}_{2}=\ell^{2}$, with the standard orthonormal basis $\left(
\mathbf{e}_{n}\right)  _{n=1}^{\infty}$, the question essentially asks
whether an infinite matrix $A=\left[  a_{mn}\right]  _{m,n=1}^{\infty}$
is bounded on $\ell^{2}$ when all of its rows and columns are uniformly in
$\ell^{2}$. For $\gamma>\frac{1}{2}$, the matrix $A$ with $a_{mn}=n^{-\gamma}$
if $n\geq m$ and $0$ otherwise, has all its rows and columns uniformly in
$\ell^{2}$, but the reader can easily compute that for $\frac{1}{2}<\gamma
\leq\frac{3}{4}$, we have $\left\Vert A\mathbf{v}\right\Vert _{\ell^{2}}=\infty$ for
$\mathbf{v}=\left(  n^{-\gamma}\right)  _{n=1}^{\infty}\in\ell^{2}$. Then the
sequence $\mathcal{B}=\left\{  A_{N}\right\}  _{N=1}^{\infty}\subset B\left(
\mathcal{H}_{1},\mathcal{H}_{2}\right)  $ of matrices $A_{N}=\left[
a_{mn}^{N}\right]  _{m,n=1}^{\infty}$, with $a_{mn}^{N}=a_{mn}$ if $n\leq N$
and $0$ otherwise, fails (\ref{controlled}). Hence we ask the following
refinement of Question (\ref{question}).

\begin{description}
\item[Basic question for orthonormal bases in Hilbert spaces] Given separable
Hilbert spaces $\mathcal{H}_{1}$ and $\mathcal{H}_{2}$ and a collection of
bounded linear operators $\mathcal{B}\subset B\left(  \mathcal{H}%
_{1},\mathcal{H}_{2}\right)  $, are there naturally occurring
orthonormal bases $\mathcal{O}_{1}$ and $\mathcal{O}_{2}$ in $\mathcal{H}_{1}$
and $\mathcal{H}_{2}$ such that the answer to question (\ref{question}) is
affirmative, i.e., (\ref{controlled}) holds?
\end{description}

More generally, for Hilbert spaces $\mathcal{H}_{1}$ and $\mathcal{H}_{2}$, we
can replace the orthonormal bases $\mathcal{O}_{1}$ and $\mathcal{O}_{2}$ with
\emph{frames} $\mathcal{F}_{1}$ and $\mathcal{F}_{2}$, and ask if the answer
is affirmative. Recall that $\left\{  f_{j}\right\}  _{j=1}^{\infty}$ is a
frame for $\mathcal{H}$ if\ there are positive constants $c,C$ such that%
\[
c\leq\frac{\sum_{j=1}^{\infty}\left\vert \left\langle x,f_{j}\right\rangle
_{\mathcal{H}}\right\vert ^{2}}{\left\Vert x\right\Vert _{\mathcal{H}}^{2}%
}\leq C,\ \ \ \ \ \text{for all }0\neq x\in\mathcal{H}.
\]
Note that any finite union of orthonormal bases is a frame.

\begin{description}
\item[Basic question for frames in Banach spaces] Given separable Banach
spaces $\mathcal{X}_{1}$ and $\mathcal{X}_{2}$ and a collection of bounded
linear operators $\mathcal{B}\subset B\left(  \mathcal{X}_{1},\mathcal{X}%
_{2}\right)  $, are there are naturally occurring frames $\mathcal{F}%
_{1}$ and $\mathcal{F}_{2}$ in $\mathcal{X}_{1}$ and $\mathcal{X}_{2}$ such
that the answer to the Banach space analogue of question (\ref{question}) is
affirmative? Namely, do we have 
\begin{equation}
\left\Vert T\right\Vert _{\mathcal{X}_{1}\rightarrow\mathcal{X}_{2}}\leq
C\left(  \left\Vert T\right\Vert _{\mathcal{F}_{1}}+\left\Vert T^{\ast
}\right\Vert _{\mathcal{F}_{2}}\right)  ,\ \ \ \ \ \text{for every }%
T\in\mathcal{B},\label{Banach frame}%
\end{equation}
where%
\begin{align*}
\left\Vert T\right\Vert _{\mathcal{X}_{1}\rightarrow\mathcal{X}_{2}} &
\equiv\sup_{\left\Vert x\right\Vert _{\mathcal{X}_{1}}=1}\left\Vert
Tx\right\Vert _{\mathcal{X}_{2}}\text{ and }\left\Vert T\right\Vert
_{\mathcal{F}_{1}}\equiv\sup_{k\in\mathbb{N}}\left\Vert Te_{k}^{1}\right\Vert
_{\mathcal{X}_{2}}\text{ and }\left\Vert T^{\ast}\right\Vert _{\mathcal{F}%
_{2}}\equiv\sup_{k\in\mathbb{N}}\left\Vert T^{\ast}e_{k}^{2}\right\Vert
_{\mathcal{X}_{1}}\ ,
\end{align*}
and $\mathcal{F}_{1}   =\left\{  y_{k}^{1}\right\}  _{k=1}^{\infty
}$ and $\mathcal{F}_{2}=\left\{  y_{k}^{2}\right\}  _{k=1}^{\infty}$
are frames in $\mathcal{X}_{1}$ and $\mathcal{X}_{2}$ respectively? A frame $\mathcal{F}$ for a general Banach space $\mathcal{X}$ is actually a
triple $\left(  \left\{  y_{i}\right\}  _{i=1}^{\infty},X_{d},S\right)  $, see
Section \ref{subsection:frames_weighted_Lp} below.
\end{description}

In this paper we will fix a dyadic grid $\mathcal{D}$ in $\mathbb{R}^{n}$, and
first consider the  Hilbert spaces $\mathcal{H}_{1}=L^{2}\left(  \sigma\right)
,\mathcal{H}_{2}=L^{2}\left(  \omega\right)  $ with respective orthonormal
bases $\mathcal{O}_{1}=\left(  h_{I}^{\sigma}\right)  _{I\in\mathcal{D}%
},\mathcal{O}_{2}=\left(  h_{J}^{\omega}\right)  _{J\in\mathcal{D}}$, where
$\sigma$ and $\omega$ are doubling measures on $\mathbb{R}^{n}$, and $\left\{
h_{I}^{\mu}\right\}  _{I\in\mathcal{D}}$ is the classical weighted Haar
wavelets on $\mathbb{R}^{n}$ associated to a measure $\mu$. In our main Theorem \ref{thm:main_L2} in Section \ref{section:Haar_results} below, we show that the
class $\mathcal{B}_{CZ}$ of truncated gradient elliptic $\lambda$-fractional
smooth Calder\'{o}n-Zygmund operators $T^{\lambda}$, $0\leq\lambda<n$, which is of course
contained in $B\left(  \mathcal{H}_{1},\mathcal{H}_{2}\right)  $, satisfies (\ref{controlled}). Recall that when $\lambda=0$, this class of
operators includes truncations of the classical Hilbert, Riesz and Beurling transforms.
This not only gives an affirmative answer to question (\ref{question}) in this
case, it also shows that any gradient-elliptic $\lambda$-fractional smooth
Calder\'{o}n-Zygmund operator $T$ is bounded from $L^{2}\left(  \sigma\right)
$ to $L^{2}\left(  \omega\right)  $ (in the sense of Section \ref{Sec CZ}) \emph{if and
only if}
\[
\left\Vert T\right\Vert _{\mathcal{O}_{1}}+\left\Vert T^{\ast}\right\Vert
_{\mathcal{O}_{2}}<\infty.
\]

Then in Section \ref{subsection:frames_weighted_Lp}, we consider Banach spaces $\mathcal{X}_{1}=L^{p}\left(  \sigma\right)
,\mathcal{X}_{2}=L^{p}\left(  \omega\right)  $ for some $1<p<\infty$, and in
Theorem \ref{thm:Lp_Haar_Ap_quad} demonstrate the analogous theorem in this
setting, namely that the class $\mathcal{B}_{CZ}$ of truncated gradient
elliptic $\lambda$-fractional smooth Calder\'{o}n-Zygmund operators $T^{\lambda}$,
$0\leq\lambda<n$, is controlled by any fixed pair of frames $\mathcal{F}%
_{1}=\left\{  \frac{h_{I}^{\sigma}}{\left\Vert h_{I}^{\sigma}\right\Vert
_{L^{p}\left(  \sigma\right)  }}\right\}  _{I\in\mathcal{D}}$ and
$\mathcal{F}_{2}=\left\{  \frac{h_{I}^{\omega}}{\left\Vert h_{I}^{\omega
}\right\Vert _{L^{p^{\prime}}\left(  \sigma\right)  }}\right\}  _{I\in
\mathcal{D}}$ for $L^{p}\left(  \sigma\right)  $ and $L^{p^{\prime}}\left(
\omega\right)  $ respectively, i.e. (\ref{Banach frame}) holds, so long as the
measure pair $(\sigma,\omega)$ satisfies the quadratic, or vector-valued,
Muckenhoupt condition $A_{p}^{\ell^{2},\operatorname{offset}}(\sigma
,\omega)<\infty$, where $A_{p}^{\ell^{2},\operatorname{offset}}(\sigma
,\omega)$ is defined as in (\ref{quad A2 tailless}). In Theorem
\ref{theorem:Lp_Haar_quad}, we are able to get rid of the condition
$A_{p}^{\ell^{2},\operatorname{offset}}(\sigma,\omega)<\infty$, but in return
must replace the scalar testing characteristics by their vector-valued analogues.

Finally we extend Theorems \ref{thm:main_L2} and \ref{theorem:Lp_Haar_quad} to Theorems \ref{main' Alpert} and \ref{main' p glob Alpert}, respectively, by replacing the weighted Haar wavelets by weighted Alpert wavelets, which form an orthonormal basis for $L^p$ when $p=2$ and a frame when $1<p<\infty$.

Beyond the fact that the main Theorem \ref{thm:main_L2}
is a nice, clean functional-analytic statement, Theorem \ref{thm:main_L2} is
especially significant because a decomposition of functions into linear
combinations of orthogonal Haar wavelets plays a vital role in the proofs of
the two-weight $T1$ theorems for Calder\'{o}n-Zygmund operators. Theorem \ref{thm:main_L2} suggests one of the reasons Haar functions play such
a big role in the two-weight $T1$ theorems (where one tests $T^{\lambda}$ against indicators of intervals rather than Haar functions) is because the Haar functions
themselves typically satisfy a $Th$ theorem, i.e. the operator
$T^{\lambda}$ is bounded if and only if the $T^{\lambda}$ satisfies a Haar
testing condition. Similarly, Theorem \ref{main' Alpert} is significant because weighted Alpert wavelets play an analogous role in the recent proofs of the two-weight $T1$ theorems for doubling measures \cite{RaSaWi, AlSaUr, SaWi}.

\section{Preliminaries: Calder\'on-Zygmund operators and weighted Haar wavelets}

\subsection{The operators: $\lambda$-fractional Calder\'{o}n-Zygmund
operators\label{Sec CZ}}

The class of bounded operators we consider are the
admissible truncations of smooth $\lambda$-fractional Calder\'{o}n-Zygmund
operators $T^{\lambda}$ on $\mathbb{R}^{n}$ for $0\leq\lambda<n$, as defined
for example in \cite[Section 2.1.2]{AlSaUr}, \cite{SaShUr9}. Such operators are associated
with real, smooth kernels $K^{\lambda}\left(  x,y\right)  $ satisfyingthe Calder\'on-Zygmund size and smoothness estimates
\begin{align} \label{size and smoothness}
  \left\vert \nabla_{x}^{m}K^{\lambda}\left(  x,y\right)  \right\vert
+\left\vert \nabla_{y}^{m}K^{\lambda}\left(  x,y\right)  \right\vert \leq
C_{\operatorname*{CZ}}\left\vert x-y\right\vert ^{\lambda-n-m} \, , \quad \text{for }m\geq0,\ \text{and }x,y\in\mathbb{R}^{n}\text{
with}\ x\neq y \, .
\end{align}
In particular, given a kernel $K^{\lambda}$, we may define smooth
\emph{admissible} truncations $K_{\varepsilon,R}^{\lambda}\left(  x,y\right)
$ which are constructed so that they vanish on $\{(x,y)\in\mathbb{R}%
^{n}\times\mathbb{R}^{n}~:~|x-y|<\varepsilon\text{ or }|x-y|>R\}$, are smooth,
and continue to satisfy (\ref{size and smoothness}). We identify
the collection of admissible truncations $K_{\varepsilon,R}^{\lambda}\left(
x,y\right)  $ of the kernel $K^{\lambda}$ with the operator $T^{\lambda}$, and we say that $T^{\lambda}$ is bounded from
$L^{p}\left(  \sigma\right)  $ to $L^{p^{\prime}}\left(  \omega\right)  $ if there exists a constant $\mathfrak{N}_{T^{\lambda},p}\left(  \sigma,\omega\right)$, referred to as the norm of $T^{\lambda}$, such that for all $f \in L^p \left (\sigma \right )$, we have
\begin{align*}
\left(  \int_{\mathbb{R}^{n}}\left\vert \int_{\mathbb{R}^{n}}K_{\varepsilon
,R}^{\lambda}\left(  x,y\right)  f\left(  y\right)  d\sigma\left(  y\right)
\right\vert ^{p}d\omega\left(  x\right)  \right)  ^{\frac{1}{p}} &
\leq\mathfrak{N}_{T^{\lambda},p}\left(  \sigma,\omega\right)  \left(
\int_{\mathbb{R}^{n}}\left\vert f\left(  y\right)  \right\vert ^{p}%
d\sigma\left(  y\right)  \right)  ^{\frac{1}{p}}\, .
\end{align*}

See, e.g., \cite{SaShUr9} for more details on this, and see \cite{Saw6} for how such bounds imply the
existence of a bounded operator that agrees with the truncations, extending the Lebesgue measure case in \cite{Ste2}.



Let $\kappa\in\mathbb{Z}_{+}$. We say that $T^{\lambda}$ is a $\kappa
$\emph{-elliptic} Calder\'{o}n-Zygmund operator on $\mathbb{R}^{n}$ if in
addition to (\ref{size and smoothness}), there is a unit vector $\mathbf{v}%
\in\mathbb{S}^{n-1}$ and $c_{1}>0$ such that the kernel $K^{\lambda}\left(
x,y\right)  $ of $T^{\lambda}$ satisfies,
\begin{equation}
\label{eq:gradient_elliptic}\left\vert \left(
\frac{d}{dt}\right)  ^{\kappa}K^{\lambda}\left(  x,x+t\mathbf{v}\right)
\right\vert \geq c_{1}\left | t \right |^{\lambda-n-\kappa},\ \ \ \ \ \text{for all }%
x\in\mathbb{R}^{n}\text{ and }t>0\text{.}%
\end{equation}
Note that by the change of variables $y = x+ t \mathbf{v}$ and replacing $t$ by $-t$, we have \eqref{eq:gradient_elliptic} is equivalent to  
\[
\left\vert \left(  \frac{d}{dt}\right)  ^{\kappa
}K^{\lambda}\left(  y+t\mathbf{v},y\right)  \right\vert \geq c_{1}\left | t \right |^{\lambda-n-\kappa} \, , \quad \text{for all }%
y \in\mathbb{R}^{n}\text{ and }t>0\text{.}%
\, . 
\]
If $\kappa=0$, then $T^{\lambda}$ is called \emph{Stein elliptic}. If $\kappa=1$, then $T^{\lambda}$ is called \emph{gradient-elliptic}, as introduced in \cite{SaShUr10}. One easily checks
that in the case $\kappa=1$, both functions
\[
\varphi_{y,\mathbf{v}}\left(  t\right)  =K^{\lambda}\left(  y+t\mathbf{v}%
,y\right)  \text{ and }\psi_{x,\mathbf{v}}\left(  t\right)  =K^{\lambda
}\left(  x,x+t\mathbf{v}\right)
\]
are monotone on $\left(  0,\infty\right)  $ and $\left(  -\infty,0\right)  $,
and that
\[
\left\vert \varphi_{y,\mathbf{v}}^{\prime}\left(  t\right)  \right\vert
\approx\left\vert \psi_{y,\mathbf{v}}^{\prime}\left(  t\right)  \right\vert
\approx\left\vert t\right\vert ^{\lambda-n-1},\ \ \ \ \ \text{for all }%
x,y\in\mathbb{R}^{n}\,\ \text{and }t\in\mathbb{R\setminus}\left\{  0\right\}
.
\]
Finally, we point out that $\kappa$-ellipticity implies $\kappa^{\prime}%
$-ellipticity for all $0\leq\kappa^{\prime}\leq\kappa$ upon using the
fundamental theorem of calculus on infinite rays, together with the size and
smoothness bounds (\ref{size and smoothness}).

We will often make reference to the \emph{Calder\'on-Zygmund data} for an operator $T^{\lambda}$: this refers to the constants in (\ref{size and smoothness}) and, depending on the context, the constant in \eqref{eq:gradient_elliptic}. 

\subsection{Orthonormal bases and weighted $L^{2}$ spaces}

For us, a cube will always be axis-parallel unless otherwise specified. We let $\mathcal{P}^{n}$ denote the set of all axis-parallel cubes in $\mathbb{R}^{n}$.

Recall that a measure $\mu$ on $\mathbb{R}^{n}$ is doubling if there exists a
constant $C>0$ such that 
\begin{equation}\label{eq:doubling}
	\mu\left(  2 Q \right)  \leq C \mu\left(  Q \right) \quad \text{ for all cubes } Q \, ,
\end{equation}
where $2 Q$ is the cube with same center but with sidelength double that of $Q$. The best constant $C$ in \eqref{eq:doubling} is the
\emph{doubling constant} for $\mu$.

A tiling $\mathcal{T}$ of $\mathbb{R}^{n}$ is a collection
of cubes $Q$ whose interiors are pairwise disjoint, and whose union is
$\mathbb{R}^{n}$. A dyadic grid $\mathcal{D}$ is a union
$\bigcup\limits_{k \in\mathbb{Z}} \mathcal{D}_{k}$ of tilings $\mathcal{D}%
_{k}$ of $\mathbb{R}^{n}$ satisfy the
following additional conditions

\begin{itemize}
	\item all cubes in $\mathcal{D}_{k}$ have sidelength $2^{-k}$.

\item any cube $Q \in\mathcal{D}_{k+1}$ is contained in a cube $Q^{\prime}%
\in\mathcal{D}_{k}$, called the dyadic parent of $Q$.

\item every cube $Q^{\prime}\in\mathcal{D}_{k}$ contains  $2^{n}$ cubes 
in $\mathcal{D}_{k+1}$, called its dyadic children.
\end{itemize}

The standard dyadic grid on $\mathbb{R}$ is the collection
of intervals $\left[  j2^{-k},\left(  j+1\right)  2^{-k}\right)  $ for
$j,k\in\mathbb{Z}$. Given a measure $\mu$ on $\mathbb{R}^{n}$ and an interval
$I$, a $\mu$-weighted Haar wavelet $h_{I}^{\mu}$ on $I$ is an $L^2 \left (\mu \right )$-normalized 
function satisfying the following properties:

\begin{itemize}
\item $\operatorname*{Supp}h_{I}^{\mu}$ is a union of dyadic children of $I$.

\item $h_{I}^{\mu}$ is constant on each dyadic child of $I$.

\item $\int h_{I}^{\mu}d\mu=0$.
\end{itemize}

If $I \subset \mathbb{R}^n$, the linear span of all the $\mu
$-weighted Haar wavelets on $I$ has dimension $2^{n}-1$. Hence when $n=1$, a Haar
wavelet $h_{I}^{\mu}$ on $I$ is determined uniquely up to its sign. But when $n \geq 2$, there many choices of $L^{2}(\mu
)$-orthonormal vectors $\{h_{I}^{\mu}\}$ which span this space. We let $\left\{  h_{I}^{\mu,\gamma
}\right\}  _{\gamma\in\Gamma_{I,n}^{\mu}}$ denote a choice of orthonormal
basis for this space, where $\Gamma_{I,n}^{\mu}$ is an index set of size equal to the dimension $2^{n}-1$.
But often we will be imprecise about which specific Haar wavelet we are considering, and will simply write
$h_{I}^{\mu}$ to refer to \emph{some} $\mu$-weighted Haar wavelet on 
$I$; similarly $\left\{  h_{I}^{\mu}\right\}  $ refers to the
collection of $\mu$-weighted Haar wavelets on $I$.

If a measure $\mu$ is doubling, then $\left\{  h_{I}^{\mu}\right\}
_{I\in\mathcal{D}}$ is an orthonormal basis of $L^{2}\left(  \mu\right)  $.
Given a dyadic grid $\mathcal{D}$,$\ $define the global and local
$\mathcal{D}$-Haar testing characteristics for $T^{\lambda}$ by%
\begin{align*}
\mathfrak{H}_{T^{\lambda}}^{\mathcal{D},\operatorname*{glob}}\left(
\sigma,\omega\right)   &  \equiv\sup_{I\in\mathcal{D}} \sup\limits_{ \left\{
h_{I}^{\sigma}\right\}  } \left\Vert T_{\sigma}^{\lambda}h_{I}^{\sigma
}\right\Vert _{L^{2}\left(  \omega\right)  }\text{ and }\mathfrak{H}%
_{T^{\lambda}}^{\mathcal{D},\operatorname*{glob}}\left(  \omega,\sigma\right)
\equiv\sup_{I\in\mathcal{D}} \sup\limits_{\left\{  h_{I}^{\omega}\right\}
}\left\Vert T_{\omega}^{\lambda,\ast}h_{I}^{\omega}\right\Vert _{L^{2}\left(
\sigma\right)  },\\
\mathfrak{H}_{T^{\lambda}}^{\mathcal{D},\operatorname{loc}}\left(
\sigma,\omega\right)   &  \equiv\sup_{I\in\mathcal{D}} \sup\limits_{ \left\{
h_{I}^{\sigma}\right\}  }\left\Vert \mathbf{1}_{I}T_{\sigma}^{\lambda}%
h_{I}^{\sigma}\right\Vert _{L^{2}\left(  \omega\right)  }\text{ and
}\mathfrak{H}_{T^{\lambda}}^{\mathcal{D},\operatorname{loc}}\left(
\omega,\sigma\right)  \equiv\sup_{I\in\mathcal{D}} \sup\limits_{ \left\{
h_{I}^{\omega}\right\}  }\left\Vert \mathbf{1}_{I}T_{\omega}^{\lambda}%
h_{I}^{\omega}\right\Vert _{L^{2}\left(  \sigma\right)  },
\end{align*}
and similarly for the familiar global, triple and local cube testing
characteristics for $T^{\lambda}$,%
\begin{align*}
\mathfrak{T}_{T^{\lambda}}^{\mathcal{D},\operatorname*{glob}}\left(
\sigma,\omega\right)   &  \equiv\sup_{I\in\mathcal{D}}\left\Vert T_{\sigma
}^{\lambda}\frac{\mathbf{1}_{I}}{\sqrt{\left\vert I\right\vert _{\sigma}}%
}\right\Vert _{L^{2}\left(  \omega\right)  }\text{ and }\mathfrak{T}%
_{T^{\lambda}}^{\mathcal{D},\operatorname*{glob}}\left(  \omega,\sigma\right)
\equiv\sup_{I\in\mathcal{D}}\left\Vert T_{\omega}^{\lambda,\ast}%
\frac{\mathbf{1}_{I}}{\sqrt{\left\vert I\right\vert _{\omega}}}\right\Vert
_{L^{2}\left(  \sigma\right)  },\\
\mathfrak{T}_{T^{\lambda}}^{\mathcal{D},\operatorname*{trip}}\left(
\sigma,\omega\right)   &  \equiv\sup_{I\in\mathcal{D}}\left\Vert T_{\sigma
}^{\lambda}\frac{\mathbf{1}_{I}}{\sqrt{\left\vert I\right\vert _{\sigma}}%
}\right\Vert _{L^{2}\left(  \omega\right)  }\text{ and }\mathfrak{T}%
_{T^{\lambda}}^{\mathcal{D},\operatorname*{trip}}\left(  \omega,\sigma\right)
\equiv\sup_{I\in\mathcal{D}}\left\Vert T_{\omega}^{\lambda,\ast}%
\frac{\mathbf{1}_{I}}{\sqrt{\left\vert I\right\vert _{\omega}}}\right\Vert
_{L^{2}\left(  \sigma\right)  },\\
\mathfrak{T}_{T^{\lambda}}^{\mathcal{D},\operatorname{loc}}\left(
\sigma,\omega\right)   &  \equiv\sup_{I\in\mathcal{D}}\left\Vert
\mathbf{1}_{I}T_{\sigma}^{\lambda}\frac{\mathbf{1}_{I}}{\sqrt{\left\vert
I\right\vert _{\sigma}}}\right\Vert _{L^{2}\left(  \omega\right)  }\text{ and
}\mathfrak{T}_{T^{\lambda}}^{\mathcal{D},\operatorname{loc}}\left(
\omega,\sigma\right)  \equiv\sup_{I\in\mathcal{D}}\left\Vert \mathbf{1}%
_{I}T_{\omega}^{\lambda,\ast}\frac{\mathbf{1}_{I}}{\sqrt{\left\vert
I\right\vert _{\omega}}}\right\Vert _{L^{2}\left(  \sigma\right)  }.
\end{align*}


Finally, we define the fractional Muckenhoupt characteristic%
\begin{equation}
\label{eq:A2}A_{2}^{\lambda}\left(  \sigma,\omega\right)  \equiv\sqrt
{\sup_{I\in\mathcal{P}^{n}}\frac{\left\vert I\right\vert _{\sigma}}{\left\vert
I\right\vert ^{1-\frac{\lambda}{n}}}\frac{\left\vert I\right\vert _{\omega}%
}{\left\vert I\right\vert ^{1-\frac{\lambda}{n}}}},\ \ \ \ \ 0\leq\lambda<n \, .
\end{equation}

\section{Main results: Haar wavelets}\label{section:Haar_results}
If $f$ and $g$ are two nonnegative functions, we write $f \lesssim g$ if there exists a positive constant $C$ such that
\[
f \leq C g \, ,
\]
for all instances of the arguments of $f, g$. We define $f \gtrsim g$ similarly, and write $f \approx g$ if $f \lesssim g$ and $f \gtrsim g$.

Throughout this paper, the  constants implicit  in $\lesssim, \gtrsim$ and
$\approx$ only depend on the doubling constants for
$\sigma,\omega$, the Calder\'{o}n-Zgymund data for the 
operator $T^{\lambda}$, $n$, and also on $\kappa$ and $p$ when they appear. Our main theorem is the following.
\begin{theorem}
\label{thm:main_L2} \label{main'}If $\sigma$ and $\omega$ are doubling
measures on $\mathbb{R}^{n}$, $\mathcal{D}$ is a dyadic grid on $\mathbb{R}%
^{n}$, $0\leq\lambda<n$, and $T^{\lambda}$ is a smooth gradient-elliptic Calder\'{o}n-Zygmund
operator on $\mathbb{R}^{n}$, then%
\[
\mathfrak{N}_{T^{\lambda}}\left(  \sigma,\omega\right)  \approx\mathfrak{H}%
_{T^{\lambda}}^{\mathcal{D},\operatorname*{glob}}\left(  \sigma,\omega\right)
+\mathfrak{H}_{T^{\lambda,\ast}}^{\mathcal{D},\operatorname*{glob}}\left(
\omega,\sigma\right)  \ .
\]

\end{theorem}

To prove this theorem, we begin with the following result, whose proof encapsulates the key ideas of this paper.

\begin{theorem}
\label{Haar}If $\sigma$ and $\omega$ are doubling measures on $\mathbb{R}^{n}%
$, $\mathcal{D}$ is a dyadic grid on $\mathbb{R}^{n}$, $0\leq\lambda<n$, and
$T^{\lambda}$ is a gradient-elliptic $\lambda$-fractional Calder\'{o}n-Zygmund
operator, then%
\[
\mathfrak{T}_{T^{\lambda}}^{\mathcal{D},\operatorname*{trip}}\left(
\sigma,\omega\right)  +A_{2}^{\lambda}\left(  \sigma,\omega\right)
\lesssim\mathfrak{H}_{T^{\lambda}}^{\mathcal{D},\operatorname*{glob}}\left(
\sigma,\omega\right)  \ .
\]

\end{theorem}

\begin{proof}
We first consider the $A_{2}^{\lambda}\left(  \sigma,\omega\right)  $
characteristic. Let $\mathbf{v}$ be a unit vector for which the
gradient-ellipticity condition (\ref{eq:gradient_elliptic}) holds for our
operator $T^{\lambda}$. Then there exists $\delta_{0}>0$, only depending on the constant in \eqref{eq:gradient_elliptic} such that
\begin{equation}
\left\vert \frac{\partial}{\partial t}K^{\lambda}\left(  x,x+t\mathbf{w}%
\right)  \right\vert =\left\vert \psi_{x,\mathbf{w}}^{\prime}\left(  t\right)
\right\vert \approx\left\vert t\right\vert ^{\lambda-n-1},\ \ \ \ \ \text{for
all}\left\vert \mathbf{w}-\mathbf{v}\right\vert <\delta_{0}.\label{perturb}%
\end{equation}
Indeed, first we have
\begin{align*}
&  \frac{\partial}{\partial t}K^{\lambda}\left(  x,x+t\mathbf{w}\right)
-\frac{\partial}{\partial t}K^{\lambda}\left(  x,x+t\mathbf{v}\right)
=\left(  \mathbf{w}\cdot\nabla_{2}\right)  K^{\lambda}\left(  x,x+t\mathbf{w}%
\right)  -\left(  \mathbf{v}\cdot\nabla_{2}\right)  K^{\lambda}\left(
x,x+t\mathbf{v}\right)  \\
&  =\left(  \left(  \mathbf{w}-\mathbf{v}\right)  \cdot\nabla_{2}\right)
K^{\lambda}\left(  x,x+t\mathbf{w}\right)  +\left(  \mathbf{v}\cdot\nabla
_{1}\right)  \left(  K^{\lambda}\left(  x,x+t\mathbf{w}\right)  -K^{\lambda
}\left(  x,x+t\mathbf{v}\right)  \right)  .
\end{align*}
By the Calder\'{o}n-Zygmund size and smoothness conditions 
(\ref{size and smoothness}), we have
\begin{align*}
\left\vert \left(  \left(  \mathbf{w}-\mathbf{v}\right)  \cdot\nabla
_{2}\right)  K^{\lambda}\left(  x,x+t\mathbf{w}\right)  \right\vert  &  \leq
C_{\operatorname*{CZ}}\left\vert \mathbf{w}-\mathbf{v}\right\vert \left\vert
t\right\vert ^{\lambda-n-1},\\
\left\vert \left(  \mathbf{v}\cdot\nabla_{1}\right)  \left(  K^{\lambda
}\left(  x,x+t\mathbf{w}\right)  -K^{\lambda}\left(  x,x+t\mathbf{v}\right)
\right)  \right\vert  &  \leq C_{\operatorname*{CZ}}\left\vert t\mathbf{w}%
-t\mathbf{v}\right\vert \left\vert t\right\vert ^{\lambda-n-2}.
\end{align*}
Thus by gradient ellipticity, we get
\[
\left\vert \frac{\partial}{\partial t}K^{\lambda}\left(  x,x+t\mathbf{w}%
\right)  -\frac{\partial}{\partial t}K^{\lambda}\left(  x,x+t\mathbf{v}%
\right)  \right\vert \leq C_{\operatorname*{CZ}}\left\vert \mathbf{w}%
-\mathbf{v}\right\vert \left\vert t\right\vert ^{\lambda-n-1}\leq\frac{1}%
{2}\left\vert \frac{\partial}{\partial t}K^{\lambda}\left(  x,x+t\mathbf{v}%
\right)  \right\vert ,
\]
for $\left\vert \mathbf{w}-\mathbf{v}\right\vert <\delta_{0}$ with $\delta
_{0}>0$ sufficiently small, and (\ref{perturb}) follows.

Now given $\delta>0$, let $S\left(  \mathbf{v},\delta\right)  $ denote the
conical sector
\[
S\left(  \mathbf{v},\delta\right)  \equiv\left\{  z\in\mathbb{R}%
^{n}:\left\vert \frac{z}{\left\vert z\right\vert }-\mathbf{v}\right\vert
<\delta\right\}  \text{.}%
\]
Let $0<\delta\leq\delta_{0}$ be a small constant that we will fix later. Then
there is $m=m(\delta)\in\mathbb{N}$ with the following property. Given dyadic
cubes $I$ and $J$ of equal side length $\ell\left(  I\right)  $ with
$\operatorname*{dist}\left(  I,J\right)  \approx\frac{1}{\delta}\ell\left(
I\right)  $, and centers $c_{I}$ and $c_{J}$, such that $J\subset
c_{I}+S\left(  \mathbf{v},\delta\right)  $, then there exist two dyadic cubes
$K,L\in\mathfrak{C}_{\mathcal{D}}^{\left(  m\right)  }\left(  I\right)  $ such
that $\operatorname*{dist}\left(  3K,3L\right)  \approx\ell\left(  I\right)  $
and $L\subset c_{K}+S\left(  \mathbf{v},\delta\right)  $. See Figure
\ref{cubes}.%
\begin{figure}[ht] 
  \fbox{\includegraphics[width=0.75\linewidth]{Haar_testing_picture.png}}
	\caption{Cubes $I, J$ (and $K,L$) configured along the conical sector $S(\mathbf{v}, \delta) + c_I$ (or $S(\mathbf{v}, \delta) + c_K$). }
\label{cubes}
\end{figure}
Set $c\equiv\frac{c_{K}+c_{L}}{2}$ and consider
\begin{equation}
\varphi\equiv\frac{1}{\left\vert L\right\vert _{\sigma}}\mathbf{1}_{L}%
-\frac{1}{\left\vert K\right\vert _{\sigma}}\mathbf{1}_{K} \, .\label{def phi}
\end{equation}
Since $\varphi$ has $\sigma$-mean zero, i.e., $\int\varphi\,d\sigma=0$, then for $x\in J$ we have%
\begin{align}
&  T_{\sigma}^{\lambda}\varphi\left(  x\right)  =\int_{I}K^{\lambda}\left(
x,y\right)  \left(  \frac{1}{\left\vert L\right\vert _{\sigma}}\mathbf{1}%
_{L}\left(  y\right)  -\frac{1}{\left\vert K\right\vert _{\sigma}}%
\mathbf{1}_{K}\left(  y\right)  \right)  d\sigma\left(  y\right)
\label{T phi pointwise}\\
&  =\int_{I}\left(  K^{\lambda}\left(  x,y\right)  -K^{\lambda}\left(
x,c\right)  \right)  \left(  \frac{1}{\left\vert L\right\vert _{\sigma}%
}\mathbf{1}_{L}\left(  y\right)  -\frac{1}{\left\vert K\right\vert _{\sigma}%
}\mathbf{1}_{K}\left(  y\right)  \right)  d\sigma\left(  y\right)  \, .\nonumber
\end{align}
We may assume
\begin{equation}
\label{eq:assumption}\left(  \left(  c-x \right)  \cdot\nabla_{2}\right)
K^{\lambda}\left(  x,c\right)  >0
\end{equation}
by (\ref{perturb}) since $c-x\in S\left(  \mathbf{v},\delta_{0}\right)  $ for
$\delta= \delta(\delta_{0})$ sufficiently small (the case  $<0$ is similar). Then we claim that for $x \in J$ and $y
\in L \cup K$, we have
\begin{equation}
\operatorname{sgn} \left\{  K^{\lambda}\left(  x,y\right)  -K^{\lambda}\left(
x,c\right)  \right\}  = \operatorname{sgn}\left(  y-c\right)  \cdot\mathbf{v}
\, , \quad\text{ and } \left|  K^{\lambda}\left(  x,y\right)  -K^{\lambda
}\left(  x,c\right)  \right|  \approx\frac{1}{\left\vert I\right\vert
^{1-\frac{\lambda}{n}}}. \label{sign}%
\end{equation}


To see this, first note that $\left\vert \frac{y-c}{\left\vert y-c\right\vert
}-\mathbf{v}\right\vert <C\delta$ when $y\in L$, and $\left\vert \frac
{y-c}{\left\vert y-c\right\vert }+\mathbf{v}\right\vert <C\delta$ when $y\in
K$. Note then that $\left\vert \frac{y-c}{\left\vert y-c\right\vert }-\left(
\pm\mathbf{v}\right)  \right\vert <C\delta$ implies that $\operatorname{sgn}%
(y-c)\cdot\mathbf{v}=\pm1$ for $\delta$ sufficiently small, meaning that
\[
\operatorname{sgn}(y-c)\cdot\mathbf{v}=%
\begin{cases}
+1 & \text{ if }y\in L\\
-1 & \text{ if }y\in K
\end{cases}
\,.
\]
Now let $\mathbf{w}\equiv\frac{x-c}{\left\vert x-c\right\vert }$. If
$y\in L$ and $x\in J$, then $\left\vert \frac{y-c}{\left\vert
y-c\right\vert }-\mathbf{w}\right\vert <C\delta$. Under these assumptions, we
compute
\begin{align*}
	& K^{\lambda}\left(  x,y\right)  -K^{\lambda}\left(  x,c\right) =\int_{0}^{1}\frac{\partial}{\partial t}K^{\lambda}\left(  x,c  +t\left(  y-c\right)  \right)  dt =\left\vert y-c\right\vert \int_{0}^{1}\left[  \left(  \frac{y-c}{\left\vert y-c\right\vert }\right)  \cdot\nabla_{2}K^{\lambda}\left(
x,c  +t\left(  y-c\right)  \right)  \right]  dt\\
	=&\left\vert y-c\right\vert \int_{0}^{1}\left[  \left(  \frac{y-c}%
{\left\vert y-c\right\vert }-\mathbf{w}\right)  \cdot\nabla_{2}K^{\lambda
}\left(  x,c  +t\left(  y-c\right)  \right)  \right]  dt\\
&  +\left\vert y-c\right\vert \int_{0}^{1}\mathbf{w}\cdot\left[  \nabla
_{2}K^{\lambda}\left(  x,c  +t\left(  y-c\right)  \right)
-\nabla_{2}K^{\lambda}\left(  x,c\right)  \right]  dt  +\left\vert y-c\right\vert \int_{0}^{1}\left[  \mathbf{w}\cdot\nabla
_{2}K^{\lambda}\left(  x,c\right)  \right]  dt\\
&  \equiv i+ii+iii\,.
\end{align*}
We show that $iii$ is the main term, while $i$ and $i$ are negligible
errors if $\delta$ sufficiently small. Using
Cauchy-Schwarz and the Calder\'{o}n-Zygmund size and smoothness estimates
(\ref{size and smoothness}), we see $\left | i \right|$ is at most
\[
\left\vert y-c\right\vert \left\vert \int_{0}%
^{1}\left[  \left(  \frac{y-c}{\left\vert y-c\right\vert }-\mathbf{w}\right)
\cdot\nabla_{2}K^{\lambda}\left(  x,c  +t\left(
y-c\right)  \right)  \right]  dt\right\vert \leq CC_{\operatorname*{CZ}}%
\ell\left(  I\right)  \delta \int\limits_{0} ^{1} \frac{1}{\left\vert \left(  c-x\right)  +t\left(
y-c\right)  \right\vert ^{n-\lambda+1}} \,  dt \, ,
\]
Because $t\in\lbrack0,1]$, then 
\begin{equation}\label{eq:comp_int}
	\left\vert \left(
c-x\right)  +t\left(  y-c\right)  \right\vert \approx\left\vert c-x\right\vert
\approx\frac{1}{\delta}\ell\left(  I\right) \, , 
\end{equation}
and so 
\[
\left\vert i\right\vert \leq CC_{\operatorname*{CZ}}\delta^{n-\lambda+2}%
\frac{1}{\ell\left(  I\right)  ^{n-\lambda}}\,.
\]
To estimate $ii$, we compute
\begin{align*}
	&  \left\vert \nabla_{2}K^{\lambda}\left(  x,c+t\left ( y-c \right )
\right)  -\nabla_{2}K^{\lambda}\left(  x,c\right)  \right\vert =\left\vert
	\int_{0}^{t}\frac{d}{ds}\nabla_{2}K^{\lambda}\left(  x,c+s \left (y-c\right)  \right)  ds\right\vert \\
&  \leq\left\vert y-c\right\vert \int_{0}^{t}\left\vert \nabla_{2}%
^{2}K^{\lambda}\left(  x,c+s \left (y-c\right)  \right)  \right\vert
ds\leq C_{\operatorname*{CZ}}\left\vert y-c\right\vert \int_{0}^{t}\frac
{ds}{\left\vert  c-x +s \left (y-c\right)\right\vert ^{n-\lambda+2}}\leq CC_{\operatorname*{CZ}%
}\ell\left(  I\right)  \frac{\delta^{n-\lambda+2}}{\ell\left(  I\right)
^{n-\lambda+2}}\,
\end{align*}
where in the last step we used \eqref{eq:comp_int} with $t$ replaced by $s$. Hence $|ii|$ is at most 
\begin{align*}
 \left\vert y-c\right\vert \left\vert \int_{0}^{1}\mathbf{w}%
\cdot\left[  \nabla_{2}K^{\lambda}\left(  x,c  +t\left(
y-c\right)  \right)  -\nabla_{2}K^{\lambda}\left(  x,c\right)  \right]
dt\right\vert \leq C\ell\left(  I\right)  CC_{\operatorname*{CZ}}\ell\left(  I\right)
\frac{\delta^{n-\lambda+2}}{\ell\left(  I\right)  ^{n-\lambda+2}} =CC_{\operatorname*{CZ}}\frac{\delta^{n-\lambda+2}}{\ell\left(  I\right)
^{n-\lambda}}\,.
\end{align*}
On the other hand using (\ref{eq:gradient_elliptic}), we estimate
$\left\vert iii\right\vert $ from below by%
\begin{align*}
&  \left\vert y-c\right\vert \int_{0}^{1}\left[  \mathbf{w}\cdot\nabla
_{2}K^{\lambda}\left(  x,c\right)  \right]  dt=\left\vert y-c\right\vert
\left[  \mathbf{w}\cdot\nabla_{2}K^{\lambda}\left(  x,c\right)  \right] =\left\vert y-c\right\vert \frac{d}{dt}K^{\lambda}\left(  x,x+t\mathbf{w}%
	\right)  \mid_{t=-\left\vert x-c\right\vert } \\ &\geq C_{\operatorname{grad}}%
\ell\left(  I\right)  \frac{1}{\left\vert x-c\right\vert ^{n-\lambda+1}}  \approx C_{\operatorname{grad}}\ell\left(  I\right)  \left(  \frac{\delta
}{\ell\left(  I\right)  }\right)  ^{n-\lambda+1}=C_{\operatorname{grad}}%
\delta^{n-\lambda+1}\left(  \frac{1}{\ell\left(  I\right)  }\right)
^{n-\lambda},
\end{align*}
since
\[
\left(  c-x\right)  \in S\left(  \mathbf{v},\delta\right)
,\ \ \ \ \ \text{for }0\leq t\leq1.
\]
Altogether then%
\begin{equation}
\left\vert i\right\vert +\left\vert ii\right\vert \leq\frac{1}{2}\left\vert
iii\right\vert \label{eq:negligible}%
\end{equation}
since%
\[
CC_{\operatorname*{CZ}}\frac{\delta^{n-\lambda+2}}{\ell\left(  I\right)
^{n-\lambda}}\leq\frac{1}{4}C_{\operatorname{grad}}\delta^{n-\lambda+1}%
\frac{1}{\ell\left(  I\right)  ^{n-\lambda}}%
\]
provided $\delta\leq\frac{1}{4}\frac{C_{\operatorname{grad}}}%
{CC_{\operatorname*{CZ}}}$. Thus the $\approx$ part of (\ref{sign}) holds if
we take $\delta=\min\left\{  \frac{1}{4}\frac{C_{\operatorname{grad}}%
}{CC_{\operatorname*{CZ}}},\delta_{0}\right\}  $. To see the equality of signs in (\ref{sign}), since $i$ and $ii$ are negligible by (\ref{eq:negligible}), then the
left side of (\ref{sign}) and $iii$ share the same sign. And (\ref{eq:assumption}) implies that $iii$ is positive, completing the proof of (\ref{sign}) when $y\in L$. The case of $y\in K$ is similar.

Thus the integrand in (\ref{T phi pointwise}) doesn't change sign, and 
$T^{\lambda}\varphi\left(  x\right)  $ is of one fixed sign for all $x\in J$.
Thus 
\begin{align}
\left\vert T^{\lambda}\varphi\left(  x\right)  \right\vert  &  =\int
_{I}\left\vert K^{\lambda}\left(  x,y\right)  -K^{\lambda}\left(  x,c\right)
\right\vert \left\vert \frac{1}{\left\vert L\right\vert _{\sigma}}%
\mathbf{1}_{L}\left(  y\right)  -\frac{1}{\left\vert K\right\vert _{\sigma}%
}\mathbf{1}_{K}\left(  y\right)  \right\vert d\sigma\left(  y\right)
\label{eq:init_nondegen}\\
&  \geq\int_{I}\left\vert K^{\lambda}\left(  x,y\right)  -K^{\lambda}\left(
x,c\right)  \right\vert \left\vert \frac{1}{\left\vert L\right\vert _{\sigma}%
}\mathbf{1}_{L}\left(  y\right)  \right\vert d\sigma\left(  y\right)
\nonumber\\
&  \gtrsim\frac{1}{\ell\left(  I\right)  ^{n-\lambda}}\,,\nonumber
\end{align}
and hence
\[
\left\vert \left\langle T^{\lambda}\varphi,\mathbf{1}_{J}\right\rangle
_{\omega}\right\vert \gtrsim\frac{\left\vert J\right\vert _{\omega}%
}{\left\vert J\right\vert ^{1-\frac{\lambda}{n}}}\ .
\]


Now we note that $\frac{\varphi}{\left\Vert \varphi\right\Vert _{L^{2}\left(
\sigma\right)  }}$ is supported in $I$ with $\sigma$-mean zero, and is
constant on the dyadic $m$-grandchildren of $I$. The vector space of
such functions is the linear span of the Haar wavelets $\left\{  h_{M}%
^{\sigma,\tau}\right\}  _{M\in\mathfrak{C}_{\mathcal{D}}^{\left[  m-1\right]
}\left(  I\right)  ,\tau\in\Gamma_{M,n}^{\sigma}}$, where $\mathfrak{C}%
_{\mathcal{D}}^{\left[  m-1\right]  }\left(  I\right)  $ denotes all dyadic
grandchildren of $I$ down to level $m-1$. Since $\varphi$ belongs to this
vector space, we have the key identity,%
\begin{equation}
\frac{\varphi}{\left\Vert \varphi\right\Vert _{L^{2}\left(  \sigma\right)  }%
}=\sum_{M\in\mathfrak{C}_{\mathcal{D}}^{\left[  m-1\right]  }\left(  I\right)
,\gamma\in\Gamma_{M,n}^{\sigma}}\left\langle \frac{\varphi}{\left\Vert
\varphi\right\Vert _{L^{2}\left(  \sigma\right)  }},h_{M}^{\sigma,\gamma
}\right\rangle h_{M}^{\sigma,\gamma}\ ,\label{key identity}%
\end{equation}
and so%
\begin{align}
\left\Vert T_{\sigma}^{\lambda}\frac{\varphi}{\left\Vert \varphi\right\Vert
_{L^{2}\left(  \sigma\right)  }}\right\Vert _{L^{2}\left(  \omega\right)  } &
\leq\sum_{M\in\mathfrak{C}_{\mathcal{D}}^{\left(  m-1\right)  }\left(
I\right)  ,\gamma\in\Gamma_{M,n}^{\sigma}}\left\vert \left\langle
\frac{\varphi}{\left\Vert \varphi\right\Vert _{L^{2}\left(  \sigma\right)  }%
},h_{M}^{\sigma,\gamma}\right\rangle \right\vert \left\Vert T_{\sigma
}^{\lambda}h_{M}^{\sigma,\gamma}\right\Vert _{L^{2}\left(  \omega\right)
}\label{eq:phi_Haar_testing}\\
&  \leq\#\left\{  \left(  M,\gamma\right)  \in\mathfrak{C}_{\mathcal{D}%
}^{\left(  m-1\right)  }\left(  I\right)  \times\Gamma_{M,n}^{\sigma}\right\}
\mathfrak{H}_{T^{\lambda}}^{\mathcal{D},\operatorname*{glob}}\left(
\sigma,\omega\right)  \lesssim\mathfrak{H}_{T^{\lambda}}^{\mathcal{D}%
,\operatorname*{glob}}\left(  \sigma,\omega\right)  ,\nonumber
\end{align}
where we used Cauchy-Schwarz in the before-last inequality. Altogether we have%
\[
\frac{\left\vert J\right\vert _{\omega}}{\left\vert J\right\vert
^{1-\frac{\lambda}{n}}}\lesssim\left\vert \left\langle T_{\sigma}^{\lambda
}\varphi,\mathbf{1}_{J}\right\rangle _{\omega}\right\vert \lesssim\left\Vert
T_{\sigma}^{\lambda}\varphi\right\Vert _{L^{2}\left(  \omega\right)  }%
\sqrt{\left\vert J\right\vert _{\omega}}\lesssim\sqrt{\left\vert J\right\vert
_{\omega}}\left\Vert \varphi\right\Vert _{L^{2}\left(  \sigma\right)
}\mathfrak{H}_{T^{\lambda}}^{\mathcal{D},\operatorname*{glob}}\left(
\sigma,\omega\right)
\]
where $\left\Vert \varphi\right\Vert _{L^{2}\left(  \sigma\right)  }%
=\sqrt{\frac{1}{\left\vert L\right\vert _{\sigma}}+\frac{1}{\left\vert
K\right\vert _{\sigma}}}\approx\frac{1}{\sqrt{\left\vert I\right\vert
_{\sigma}}}$ by doubling, which gives
\[
\sqrt{\frac{\left\vert I\right\vert _{\sigma}\left\vert J\right\vert _{\omega
}}{\left\vert I\right\vert ^{1-\frac{\lambda}{n}}\left\vert J\right\vert
^{1-\frac{\lambda}{n}}}}\lesssim\mathfrak{H}_{T_{\sigma}^{\lambda}%
}^{\mathcal{D},\operatorname*{glob}}\left(  \sigma,\omega\right)  ,
\]
and since $\sigma,\omega$ are doubling, we obtain%
\[
A_{2}^{\lambda}\left(  \sigma,\omega\right)  \lesssim\mathfrak{H}_{T^{\lambda
}}^{\mathcal{D},\operatorname*{glob}}\left(  \sigma,\omega\right)  .
\]


Now we turn to the triple testing characteristic $\mathfrak{T}_{T^{\lambda}%
}^{\mathcal{D},\operatorname*{trip}}\left(  \sigma,\omega\right)  $. Fix
$L\in\mathcal{D}$ and construct $K,I\in\mathcal{D}$ so that same previous conditions are satisfied for the triple of cubes $\left(  I,K,L\right)  $. Then 
\begin{align*}
\int_{3L}\left\vert T_{\sigma}^{\lambda}\varphi\right\vert ^{2}d\omega &
=\int_{3L}\left\vert \frac{1}{\left\vert L\right\vert _{\sigma}}T_{\sigma
}^{\lambda}\mathbf{1}_{L}-\frac{1}{\left\vert K\right\vert _{\sigma}}%
T_{\sigma}^{\lambda}\mathbf{1}_{K}\right\vert ^{2}d\omega\\
&  =\int_{3L}\left\vert \frac{1}{\left\vert L\right\vert _{\sigma}}T_{\sigma
}^{\lambda}\mathbf{1}_{L}\right\vert ^{2}d\omega+\int_{3L}\left\vert \frac
{1}{\left\vert K\right\vert _{\sigma}}T_{\sigma}^{\lambda}\mathbf{1}%
_{K}\right\vert ^{2}d\omega-2\frac{1}{\left\vert L\right\vert _{\sigma
}\left\vert K\right\vert _{\sigma}}\int_{3L}\left(  T_{\sigma}^{\lambda
}\mathbf{1}_{L}\right)  \left(  T_{\sigma}^{\lambda}\mathbf{1}_{K}\right)
d\omega,
\end{align*}
where by separation of $3L$ and $3K$, and the Calder\'{o}n-Zygmund size and
smoothness estimates, we have%
\begin{align*}
\left\vert \int_{3L}\left(  T_{\sigma}^{\lambda}\mathbf{1}_{L}\right)  \left(
T_{\sigma}^{\lambda}\mathbf{1}_{K}\right)  d\omega\right\vert  &  \lesssim
\int_{3L}\left\vert T_{\sigma}^{\lambda}\mathbf{1}_{L}\right\vert
\frac{\left\vert K\right\vert _{\sigma}}{\left\vert I\right\vert
^{1-\frac{\lambda}{n}}}d\omega\lesssim\frac{\left\vert K\right\vert _{\sigma
}\sqrt{\left\vert 3L\right\vert _{\omega}}}{\left\vert I\right\vert
^{1-\frac{\lambda}{n}}}\sqrt{\int_{3L}\left\vert T_{\sigma}^{\lambda
}\mathbf{1}_{L}\right\vert ^{2}d\omega}\\
&  \lesssim\sqrt{\int_{3L}\left\vert T_{\sigma}^{\lambda}\frac{\mathbf{1}_{L}%
}{\sqrt{\left\vert L\right\vert _{\sigma}}}\right\vert ^{2}d\omega}%
\frac{\left\vert K\right\vert _{\sigma}\sqrt{\left\vert L\right\vert _{\sigma
}\left\vert 3L\right\vert _{\omega}}}{\left\vert I\right\vert ^{1-\frac
{\lambda}{n}}}\lesssim\left\vert K\right\vert _{\sigma}A_{2}^{\lambda}\left(
\sigma,\omega\right)  \sqrt{\int_{3L}\left\vert T_{\sigma}^{\lambda}%
\frac{\mathbf{1}_{L}}{\sqrt{\left\vert L\right\vert _{\sigma}}}\right\vert
^{2}d\omega}.
\end{align*}
Thus, from the previous two displays and (\ref{eq:phi_Haar_testing}) we have%
\begin{align}
&  \int_{3L}\left\vert T_{\sigma}^{\lambda}\frac{\mathbf{1}_{L}}%
{\sqrt{\left\vert L\right\vert _{\sigma}}}\right\vert ^{2}d\omega=\left\vert
L\right\vert _{\sigma}\int_{3L}\left\vert T_{\sigma}^{\lambda}\frac
{\mathbf{1}_{L}}{\left\vert L\right\vert _{\sigma}}\right\vert ^{2}%
d\omega\lesssim\left\vert L\right\vert _{\sigma}\int_{3L}\left\vert T_{\sigma
}^{\lambda}\varphi\right\vert ^{2}d\omega+\frac{2}{\left\vert K\right\vert
_{\sigma}}\int_{3L}\left\vert T_{\sigma}^{\lambda}\mathbf{1}_{L}\right\vert
\left\vert T_{\sigma}^{\lambda}\mathbf{1}_{K}\right\vert d\omega
\label{eq:init_interval_to_Haar_bd}\\
&  \lesssim\mathfrak{H}_{T^{\lambda}}^{\mathcal{D},\operatorname*{glob}%
}\left(  \sigma,\omega\right)  ^{2}\left\vert L\right\vert _{\sigma}\int
_{3L}\left\vert \varphi\right\vert ^{2}d\sigma+A_{2}^{\lambda}\left(
\sigma,\omega\right)  \sqrt{\int_{3L}\left\vert T_{\sigma}^{\lambda}%
\frac{\mathbf{1}_{L}}{\sqrt{\left\vert L\right\vert _{\sigma}}}\right\vert
^{2}d\omega}\nonumber\\
&  \lesssim\mathfrak{H}_{T^{\lambda}}^{\mathcal{D},\operatorname*{glob}%
}\left(  \sigma,\omega\right)  ^{2}+A_{2}^{\lambda}\left(  \sigma
,\omega\right)  \sqrt{\int_{3L}\left\vert T_{\sigma}^{\lambda}\frac
{\mathbf{1}_{L}}{\sqrt{\left\vert L\right\vert _{\sigma}}}\right\vert
^{2}d\omega}\,.\nonumber
\end{align}
We conclude that%
\begin{align*}
\int_{3L}\left\vert T_{\sigma}^{\lambda}\frac{\mathbf{1}_{L}}{\sqrt{\left\vert
L\right\vert _{\sigma}}}\right\vert ^{2}d\omega & \leq C\mathfrak{H}%
_{T}^{\mathcal{D},\operatorname*{glob}}\left(  \sigma,\omega\right)
^{2}+CA_{2}^{\lambda}\left(  \sigma,\omega\right)  \sqrt{\int_{3L}\left\vert
T_{\sigma}^{\lambda}\frac{\mathbf{1}_{L}}{\sqrt{\left\vert L\right\vert
_{\sigma}}}\right\vert ^{2}d\omega}\\
& \leq C\mathfrak{H}_{T}^{\mathcal{D},\operatorname*{glob}}\left(
\sigma,\omega\right)  ^{2}+\frac{C}{\varepsilon}A_{2}^{\lambda}\left(
\sigma,\omega\right)  ^{2}+C\varepsilon\int_{3L}\left\vert T_{\sigma}%
^{\lambda}\frac{\mathbf{1}_{L}}{\sqrt{\left\vert L\right\vert _{\sigma}}%
}\right\vert ^{2}d\omega,
\end{align*}
and hence by absorbing the final term on the right into the left hand side
with $\varepsilon=\frac{1}{2C}$, and then taking the supremum over $L$, we
obtain
\[
\mathfrak{T}_{T^{\lambda}}^{\mathcal{D},\operatorname*{trip}}\left(
\sigma,\omega\right)  =\sup_{L\in\mathcal{D}}\sqrt{\int_{3L}\left\vert
T_{\sigma}^{\lambda}\frac{\mathbf{1}_{L}}{\sqrt{\left\vert L\right\vert
_{\sigma}}}\right\vert ^{2}d\omega}\lesssim\mathfrak{H}_{T}^{\mathcal{D}%
,\operatorname*{glob}}\left(  \sigma,\omega\right)  +A_{2}^{\lambda}\left(
\sigma,\omega\right)  .
\]
Combining this with the previous result gives%
\[
\mathfrak{T}_{T^{\lambda}}^{\mathcal{D},\operatorname*{trip}}\left(
\sigma,\omega\right)  \lesssim\mathfrak{H}_{T^{\lambda}}^{\mathcal{D}%
,\operatorname*{glob}}\left(  \sigma,\omega\right)  .
\]

\end{proof}

In what follows, define the local cube testing characteristics
\[
\mathfrak{T}_{T^{\lambda}}^{\operatorname{loc}}\left(  \sigma,\omega\right)
\equiv\sup_{I\in\mathcal{P}^{n}}\left\Vert \mathbf{1}_{I}T_{\sigma}^{\lambda
}\frac{\mathbf{1}_{I}}{\sqrt{\left\vert I\right\vert _{\sigma}}}\right\Vert
_{L^{2}\left(  \omega\right)  }\text{ and }\mathfrak{T}_{T^{\lambda}%
}^{\operatorname{loc}}\left(  \omega,\sigma\right)  \equiv\sup_{I\in
\mathcal{P}^{n}}\left\Vert \mathbf{1}_{I}T_{\omega}^{\lambda,\ast}%
\frac{\mathbf{1}_{I}}{\sqrt{\left\vert I\right\vert _{\omega}}}\right\Vert
_{L^{2}\left(  \sigma\right)  }\, .
\]
Recall a special case of the main theorem from \cite{AlSaUr}.

\begin{theorem}
[\cite{AlSaUr}]\label{main}Suppose $0\leq\lambda<n$, and let $T^{\lambda}$ be
a $\lambda$-fractional Calder\'{o}n-Zygmund operator on
$\mathbb{R}^{n}$ with a smooth kernel $K^{\lambda}$.
Assume that $\sigma$ and $\omega$ are doubling measures on $\mathbb{R}^{n}$. Then \begin{equation}
\mathfrak{N}_{T^{\lambda}}\left(  \sigma,\omega\right)  \approx \left(  A_{2}^{\lambda}\left(  \sigma,\omega\right)  +\mathfrak{T}%
_{T^{\lambda}}^{\operatorname{loc}}\left(  \sigma,\omega\right)
+\mathfrak{T}_{\left(  T^{\lambda}\right)  ^{\ast}}^{\operatorname{loc}%
}\left(  \omega,\sigma\right)  \right)  \label{bound} \,  .
\end{equation}
\end{theorem}

In \cite[Theorem 42]{AlLuSaUr} it was shown that for any single dyadic grid
$\mathcal{D}$, we can replace the testing characteristics $\mathfrak{T}%
_{T^{\lambda}}^{\operatorname{loc}}\left(  \sigma,\omega\right)  $ and
$\mathfrak{T}_{\left(  T^{\lambda}\right)  ^{\ast}}^{\operatorname{loc}%
}\left(  \omega,\sigma\right)  $ in Theorem \ref{main} with dyadic testing
characteristics $\mathfrak{T}_{T^{\lambda}}^{\mathcal{D},\operatorname{loc}%
}\left(  \sigma,\omega\right)  $ and $\mathfrak{T}_{\left(  T^{\lambda
}\right)  ^{\ast}}^{\mathcal{D},\operatorname{loc}}\left(  \omega
,\sigma\right)  $, but the proof was quite complicated. Instead of appealing
to this result, we prove a weaker theorem that suffices for our purposes,
and with a much simpler proof, namely that we can replace the testing
characteristics with the larger \emph{triple} dyadic testing characteristics
$\mathfrak{T}_{T^{\lambda}}^{\mathcal{D},\operatorname*{trip}}\left(
\sigma,\omega\right)  $ and $\mathfrak{T}_{\left(  T^{\lambda}\right)  ^{\ast
}}^{\mathcal{D},\operatorname*{trip}}\left(  \omega,\sigma\right)  $.

\begin{proposition}
\label{single}Fix a dyadic grid $\mathcal{D}$ and assume notation as in Theorem \ref{main}. Then%
\[
\mathfrak{N}_{T^{\lambda}}\left(  \sigma,\omega\right)  \lesssim   A_{2}^{\lambda}\left(  \sigma,\omega\right)  +\mathfrak{T}%
_{T^{\lambda}}^{\mathcal{D},\operatorname*{trip}}\left(  \sigma,\omega\right)
+\mathfrak{T}_{\left(  T^{\lambda}\right)  ^{\ast}}^{\mathcal{D}%
,\operatorname*{trip}}\left(  \omega,\sigma\right)  \,   .
\]

\end{proposition}

\begin{proof}
Given a doubling measure $\mu$ on $\mathbb{R}^{n}$ and $\varepsilon>0$, a
standard construction shows that there are positive constants $0<\eta<1\leq
C_{0}$, depending only on the dimension $n$ and the doubling constant of $\mu
$, such that for any cube $I\in\mathcal{P}^{n}$, there are cubes $\left\{
J_{k}\right\}  _{k=1}^{C_{0}}\subset\mathcal{D}$ satisfying
\[
\bigcup_{k=1}^{C_{0}}J_{k}\subset\eta I\text{ and }\left\vert I\setminus
\bigcup_{k=1}^{C_{0}}J_{k}\right\vert _{\mu}<\varepsilon\left\vert
I\right\vert _{\mu}\ .
\]
Thus from Minkowski's inequality, we have%
\[
\left(  \int_{I}\left\vert T_{\sigma}^{\lambda}\mathbf{1}_{I}\right\vert
^{2}d\omega\right)  ^{\frac{1}{2}}\leq\left(  \int_{I}\left\vert T_{\sigma
}^{\lambda}\mathbf{1}_{I\setminus\bigcup_{k=1}^{C_{0}}J_{k}}\right\vert
^{2}d\omega\right)  ^{\frac{1}{2}}+\sum_{k=1}^{C_{0}}\left(  \int
_{I}\left\vert T_{\sigma}^{\lambda}\mathbf{1}_{J_{k}}\right\vert ^{2}%
d\omega\right)  ^{\frac{1}{2}},
\]
where the square of the first term on the right hand side satisfies%
\begin{align*}
\int_{I}\left\vert T_{\sigma}^{\lambda}\mathbf{1}_{I\setminus\bigcup
_{k=1}^{C_{0}}J_{k}}\right\vert ^{2}d\omega &  \leq\mathfrak{N}_{T^{\lambda}%
}\left(  \sigma,\omega\right)  ^{2}\int_{I}\left\vert \mathbf{1}%
_{I\setminus\bigcup_{k=1}^{C_{0}}J_{k}}\right\vert ^{2}d\sigma\\
&  \leq\mathfrak{N}_{T^{\lambda}}\left(  \sigma,\omega\right)  ^{2}\left\vert
I\setminus\bigcup_{k=1}^{C_{0}}J_{k}\right\vert _{\sigma}\leq\varepsilon
\mathfrak{N}_{T^{\lambda}}\left(  \sigma,\omega\right)  ^{2}\left\vert
I\right\vert _{\sigma}\ .
\end{align*}
Now we turn to estimating the squares $\int_{I}\left\vert T_{\sigma}^{\lambda
}\mathbf{1}_{J_{k}}\right\vert ^{2}d\omega$ of the remaining terms for $1\leq
k\leq C_{0}$. We have%
\begin{align*}
\int_{I}\left\vert T_{\sigma}^{\lambda}\mathbf{1}_{J_{k}}\right\vert
^{2}d\omega &  =\int_{3J_{k}}\left\vert T_{\sigma}^{\lambda}\mathbf{1}_{J_{k}%
}\right\vert ^{2}d\omega+\int_{I\setminus3J_{k}}\left\vert T_{\sigma}%
^{\lambda}\mathbf{1}_{J_{k}}\right\vert ^{2}d\omega\\
&  \leq\mathfrak{T}_{T^{\lambda}}^{\mathcal{D},\operatorname*{trip}}\left(
\sigma,\omega\right)  ^{2}\left\vert J_{k}\right\vert _{\sigma}+CA_{2}%
^{\lambda}\left(  \sigma,\omega\right)  ^{2}\left\vert J_{k}\right\vert
_{\sigma},
\end{align*}
and summing in $k$ gives%
\[
\sum_{k=1}^{C_{0}}\left(  \int_{I}\left\vert T_{\sigma}^{\lambda}%
\mathbf{1}_{J_{k}}\right\vert ^{2}d\omega\right)  ^{\frac{1}{2}}\leq
\sqrt{C_{0}}\sqrt{\sum_{k=1}^{C_{0}}\int_{I}\left\vert T_{\sigma}^{\lambda
}\mathbf{1}_{J_{k}}\right\vert ^{2}d\omega}\lesssim\left(  \mathfrak{T}%
_{T^{\lambda}}^{\mathcal{D},\operatorname*{trip}}\left(  \sigma,\omega\right)
+A_{2}^{\lambda}\left(  \sigma,\omega\right)  \right)  \sqrt{\left\vert
I\right\vert _{\sigma}}.
\]
Altogether then we have%
\[
\left(  \int_{I}\left\vert T_{\sigma}^{\lambda}\mathbf{1}_{I}\right\vert
^{2}d\omega\right)  ^{\frac{1}{2}}\lesssim\left[  \sqrt{\varepsilon
}\mathfrak{N}_{T^{\lambda}}\left(  \sigma,\omega\right)  +\mathfrak{T}%
_{T^{\lambda}}^{\mathcal{D},\operatorname*{trip}}\left(  \sigma,\omega\right)
+A_{2}^{\lambda}\left(  \sigma,\omega\right)  \right]  \sqrt{\left\vert
I\right\vert _{\sigma}},
\]
which shows that%
\[
\mathfrak{T}_{T^{\lambda}}\left(  \sigma,\omega\right)  =\sup_{I\in
\mathcal{P}^{n}}\left(  \frac{1}{\left\vert I\right\vert _{\sigma}}\int
_{I}\left\vert T_{\sigma}^{\lambda}\mathbf{1}_{I}\right\vert ^{2}%
d\omega\right)  ^{\frac{1}{2}}\lesssim\sqrt{\varepsilon}\mathfrak{N}%
_{T^{\lambda}}\left(  \sigma,\omega\right)  +\mathfrak{T}_{T^{\lambda}%
}^{\mathcal{D},\operatorname*{trip}}\left(  \sigma,\omega\right)
+A_{2}^{\lambda}\left(  \sigma,\omega\right)  .
\]
Now we conclude from this and Theorem \ref{main} that
\begin{align*}
\mathfrak{N}_{T^{\lambda}}\left(  \sigma,\omega\right)   &  \leq C_{\alpha
,n}\left(  A_{2}^{\lambda}\left(  \sigma,\omega\right)  +\mathfrak{T}%
_{T^{\lambda}}\left(  \sigma,\omega\right)  +\mathfrak{T}_{\left(  T^{\lambda
}\right)  ^{\ast}}\left(  \omega,\sigma\right)  \right) \\
&  \leq C_{\alpha,n}^{\prime}\left(  \sqrt{\varepsilon}\mathfrak{N}%
_{T^{\lambda}}\left(  \sigma,\omega\right)  +A_{2}^{\lambda}\left(
\sigma,\omega\right)  +\mathfrak{T}_{T^{\lambda}}^{\mathcal{D}%
,\operatorname*{trip}}\left(  \sigma,\omega\right)  +\mathfrak{T}_{\left(
T^{\lambda}\right)  ^{\ast}}^{\mathcal{D},\operatorname*{trip}}\left(
\omega,\sigma\right)  \right)  .
\end{align*}
Since $T^{\lambda}$ is an admissible truncation, we have $\mathfrak{N}%
_{T^{\lambda}}\left(  \sigma,\omega\right)  <\infty$, and then absorbing the
term $C_{\alpha,n}^{\prime}\sqrt{\varepsilon}\mathfrak{N}_{T^{\lambda}}\left(
\sigma,\omega\right)  $ into the left hand side with $\varepsilon<\left(
\frac{1}{2C_{\alpha,n}^{\prime}}\right)  ^{2}$, finishes the proof of
Proposition \ref{single}.
\end{proof}

We can now characterize the two weight norm inequality in terms of Haar testing alone.

\begin{proof}
[Proof of Theorem \ref{thm:main_L2}] We trivially have 
	\[
		\mathfrak{H}
_{T^{\lambda}}^{\mathcal{D},\operatorname*{glob}}\left(  \sigma,\omega\right)
 + \mathfrak{H}_{T^{\lambda,\ast}}^{\mathcal{D},\operatorname*{glob}}\left(
\omega,\sigma\right)  \lesssim \mathfrak{N}_{T^{\lambda}}\left(  \sigma
,\omega\right) \, .
	\]
Proposition \ref{single}, followed by Theorem \ref{Haar},
gives the opposite inequality,%
\begin{align*}
\mathfrak{N}_{T^{\lambda}}\left(  \sigma,\omega\right)   &  \lesssim
A_{2}^{\lambda}\left(  \sigma,\omega\right)  +\mathfrak{T}_{T^{\lambda}%
}^{\mathcal{D},\operatorname*{trip}}\left(  \sigma,\omega\right)
+\mathfrak{T}_{\left(  T^{\lambda}\right)  ^{\ast}}^{\mathcal{D}%
,\operatorname*{trip}}\left(  \omega,\sigma\right) \\
&  \lesssim\mathfrak{H}_{T^{\lambda}}^{\mathcal{D},\operatorname*{glob}%
}\left(  \sigma,\omega\right)  +\mathfrak{H}_{T^{\lambda,\ast}}^{\mathcal{D}%
,\operatorname*{glob}}\left(  \omega,\sigma\right)  .
\end{align*}

\end{proof}

\subsection{Frames and weighted $L^{p}$ spaces}

\label{subsection:frames_weighted_Lp}
We say that $\left\{  f_{j}\right\}
_{j=1}^{\infty}$ is a frame for $L^{p}\left(  \mu\right)  $ if%
\begin{equation}
c\leq\frac{\left\Vert \left(  \sum_{j=1}^{\infty}\left\vert \left\langle
f,f_{j}\right\rangle _{\mu}f_{j}\left(  x\right)  \right\vert ^{2}\right)
^{\frac{1}{2}}\right\Vert _{L^{p}\left(  \mu\right)  }}{\left\Vert
f\right\Vert _{L^{p}\left(  \mu\right)  }}\leq C,\ \ \ \ \ \text{for all
}f\neq0\in L^{p}\left(  \mu\right)  ,\label{def frame LP}%
\end{equation}
and we call the constants $c$ and $C$ the lower and upper frame bounds
respectively. For example, the collection of Haar wavelets $\left\{
h_{Q}^{\mu}\right\}  _{Q\in\mathcal{D}}$ is a frame for $L^{p}\left(
\mu\right)  $, $1<p<\infty$, since%
\[
\left\Vert \left(  \sum_{Q\in\mathcal{D}}\left\vert \bigtriangleup_{Q}^{\mu
}f\left(  x\right)  \right\vert ^{2}\right)  ^{\frac{1}{2}}\right\Vert
_{L^{p}\left(  \mu\right)  }\approx\left\Vert f\right\Vert _{L^{p}\left(
\mu\right)  }\ ,
\]
by square function estimates, where $\bigtriangleup_{Q}^{\mu}$ denotes the
projection operator onto the linear span of Haar wavelets on a cube
$Q$. The square function equivalence above follows from a modification of the proof of Burkholder's theorem \cite[Section 3.1]{SaWi}.

We note in passing that one can formulate the notion of frame in an
\emph{arbitrary} Banach space $X$ in many equivalent ways, see \cite{CaHaLa}.
For example, K. Grochenig \cite{Gro} was one of the first to generalize the
notion of frame to arbitrary Banach spaces $X$ as follows.

Given a Banach space $X$, a triple of data $\left(  \left\{  y_{i}\right\}
_{i=1}^{\infty},X_{d},S\right)  $, where

\begin{itemize}
\item $\left\{  y_{i}\right\}  _{i=1}^{\infty}\subset X^{\ast}$,

\item $X_{d}$ is a Banach space of scalar-valued sequences on $\mathbb{N}$,

\item and $S:X_{d}\rightarrow X$ is a map,
\end{itemize}

is said to be a \emph{Banach frame} for $X$ with respect to $X_{d}$, and $S$
is called the associated reconstruction operator, if the following four
properties hold:

\begin{enumerate}
\item $\left\{  \left\langle x,y_{i}\right\rangle \right\}  _{i\in\mathbb{N}%
}\in X_{d}$ for each $x\in X$,

\item The norms $\left\Vert x\right\Vert _{X}$ and $\left\Vert \left\{
\left\langle x,y_{i}\right\rangle \right\}  _{i\in\mathbb{N}}\right\Vert
_{X_{d}}$ are equivalent for $x\in X$,

\item $S$ is a linear and bounded map,

\item and $S\left(  \left\{  \left\langle x,y_{i}\right\rangle \right\}
_{i\in\mathbb{N}}\right)  =x$ for all $x\in X$.
\end{enumerate}

The reader can easily verify that the triple of data $\left(  \left\{
h_{Q}^{\mu,\gamma}\right\}  _{\substack{Q\in\mathcal{D}\\\gamma\in\Gamma
_{Q,n}^{\mu}}},X_{d},S\right)  $ is a frame for $L^{p}\left(  \mu\right)  ,$
where $X_{d}=\left\{  \left\{  a_{Q}^{\gamma}\right\}  _{\substack{Q\in
\mathcal{D}\\\gamma\in\Gamma_{Q,n}^{\mu}}}:a_{Q}^{\gamma}\in\mathbb{R}%
\right\}  $ is normed by%
\[
\left\Vert \left\{  a_{Q}^{\gamma}\right\}  _{\substack{Q\in\mathcal{D}%
\\\gamma\in\Gamma_{Q,n}^{\mu}}}\right\Vert _{X_{d}}\equiv\left\Vert \left(
\sum_{\substack{Q\in\mathcal{D}\\\gamma\in\Gamma_{Q,n}^{\mu}}}\left\vert
a_{Q}^{\gamma}h_{Q}^{\mu,\gamma}\left(  x\right)  \right\vert ^{2}\right)
^{\frac{1}{2}}\right\Vert _{L^{p}\left(  \mu\right)  },
\]
and $S:X_{d}\rightarrow X$ is defined by%
\[
S\left(  \left\{  a_{Q}^{\gamma}\right\}  _{\substack{Q\in\mathcal{D}%
\\\gamma\in\Gamma_{Q,n}^{\mu}}}\right)  \equiv\sum_{\substack{Q\in
\mathcal{D}\\\gamma\in\Gamma_{Q,n}^{\mu}}}a_{Q}^{\gamma}h_{Q}^{\mu,\gamma}.
\]


Now we restrict our attention to $L^{p}\left(  \mu\right)  $ spaces where the
simpler definition (\ref{def frame LP}) is in force. Let $1<p<\infty$ and
define the global and local Haar testing characteristics for $T^{\lambda}$ by%
\begin{align*}
\mathfrak{H}_{T^{\lambda},p}^{\operatorname*{glob}}\left(  \sigma
,\omega\right)   &  \equiv\sup_{I\in\mathcal{P}^{n}} \sup\limits_{ \left\{
h_{I} ^{\sigma} \right\}  } \frac{\left\Vert T_{\sigma}^{\lambda}h_{I}%
^{\sigma}\right\Vert _{L^{p}\left(  \omega\right)  }}{\left\Vert h_{I}%
^{\sigma}\right\Vert _{L^{p}\left(  \sigma\right)  }}\text{ and }%
\mathfrak{H}_{T^{\lambda},p}^{\operatorname*{glob}}\left(  \omega
,\sigma\right)  \equiv\sup_{I\in\mathcal{P}^{n}} \sup\limits_{ \left\{  h_{I}
^{\omega} \right\}  }\frac{\left\Vert T_{\omega}^{\lambda,\ast}h_{I}^{\omega
}\right\Vert _{L^{p^{\prime}}\left(  \sigma\right)  }}{\left\Vert
h_{I}^{\omega}\right\Vert _{L^{p^{\prime}}\left(  \omega\right)  }},\\
\mathfrak{H}_{T^{\lambda},p}^{\operatorname{loc}}\left(  \sigma,\omega\right)
&  \equiv\sup_{I\in\mathcal{P}^{n}} \sup\limits_{ \left\{  h_{I} ^{\sigma}
\right\}  }\frac{\left\Vert \mathbf{1}_{I}T_{\sigma}^{\lambda}h_{I}^{\sigma
}\right\Vert _{L^{p}\left(  \omega\right)  }}{\left\Vert h_{I}^{\sigma
}\right\Vert _{L^{p}\left(  \sigma\right)  }}\text{ and }\mathfrak{H}%
_{T^{\lambda},p}^{\operatorname{loc}}\left(  \omega,\sigma\right)  \equiv
\sup_{I\in\mathcal{P}^{n}} \sup\limits_{ \left\{  h_{I} ^{\omega} \right\}  }
\frac{\left\Vert \mathbf{1}_{I}T_{\omega}^{\lambda}h_{I}^{\omega, \gamma
}\right\Vert _{L^{p^{\prime}}\left(  \sigma\right)  }}{\left\Vert
h_{I}^{\omega}\right\Vert _{L^{p^{\prime}}\left(  \omega\right)  }},
\end{align*}
and similarly for the familiar global, triple and local cube testing
characteristics for $T^{\lambda}$,%
\begin{align*}
\mathfrak{T}_{T^{\lambda},p}^{\operatorname*{glob}}\left(  \sigma
,\omega\right)   &  \equiv\sup_{I\in\mathcal{P}^{n}}\frac{\left\Vert
T_{\sigma}^{\lambda}\frac{\mathbf{1}_{I}}{\sqrt{\left\vert I\right\vert
_{\sigma}}}\right\Vert _{L^{p}\left(  \omega\right)  }}{\left\vert
I\right\vert _{\sigma}^{\frac{1}{p}-\frac{1}{2}}}\text{ and }\mathfrak{T}%
_{T^{\lambda},p}^{\operatorname*{glob}}\left(  \omega,\sigma\right)
\equiv\sup_{I\in\mathcal{P}^{n}}\frac{\left\Vert T_{\omega}^{\lambda,\ast
}\frac{\mathbf{1}_{I}}{\sqrt{\left\vert I\right\vert _{\omega}}}\right\Vert
_{L^{p^{\prime}}\left(  \sigma\right)  }}{\left\vert I\right\vert _{\omega
}^{\frac{1}{p^{\prime}}-\frac{1}{2}}},\\
\mathfrak{T}_{T^{\lambda},p}^{\operatorname*{trip}}\left(  \sigma
,\omega\right)   &  \equiv\sup_{I\in\mathcal{P}^{n}}\frac{\left\Vert
T_{\sigma}^{\lambda}\frac{\mathbf{1}_{I}}{\sqrt{\left\vert I\right\vert
_{\sigma}}}\right\Vert _{L^{p}\left(  \omega\right)  }}{\left\vert
I\right\vert _{\sigma}^{\frac{1}{p}-\frac{1}{2}}}\text{ and }\mathfrak{T}%
_{T^{\lambda},p}^{\operatorname*{trip}}\left(  \omega,\sigma\right)
\equiv\sup_{I\in\mathcal{P}^{n}}\frac{\left\Vert T_{\omega}^{\lambda,\ast
}\frac{\mathbf{1}_{I}}{\sqrt{\left\vert I\right\vert _{\omega}}}\right\Vert
_{L^{p^{\prime}}\left(  \sigma\right)  }}{\left\vert I\right\vert _{\omega
}^{\frac{1}{p^{\prime}}-\frac{1}{2}}},\\
\mathfrak{T}_{T^{\lambda},p}^{\operatorname{loc}}\left(  \sigma,\omega\right)
&  \equiv\sup_{I\in\mathcal{P}^{n}}\frac{\left\Vert \mathbf{1}_{I}T_{\sigma
}^{\lambda}\frac{\mathbf{1}_{I}}{\sqrt{\left\vert I\right\vert _{\sigma}}%
}\right\Vert _{L^{p}\left(  \omega\right)  }}{\left\vert I\right\vert
_{\sigma}^{\frac{1}{p}-\frac{1}{2}}}\text{ and }\mathfrak{T}_{T^{\lambda}%
,p}^{\operatorname{loc}}\left(  \omega,\sigma\right)  \equiv\sup
_{I\in\mathcal{P}^{n}}\frac{\left\Vert \mathbf{1}_{I}T_{\omega}^{\lambda,\ast
}\frac{\mathbf{1}_{I}}{\sqrt{\left\vert I\right\vert _{\omega}}}\right\Vert
_{L^{p^{\prime}}\left(  \sigma\right)  }}{\left\vert I\right\vert _{\omega
}^{\frac{1}{p^{\prime}}-\frac{1}{2}}}.
\end{align*}
We may also replace $\mathcal{P}^{n}$ in the definitions above by a dyadic
grid $\mathcal{D}$ on $\mathbb{R}^{n}$ to get their $\mathcal{D}$ counterparts.

First we need an extension of Theorem \ref{main'} to $1<p<\infty$ that uses
the quadratic offset Muckenhoupt characteristic $A_{p}^{\lambda,\ell
^{2},\operatorname*{offset}}\left(  \sigma,\omega\right)  $ defined as the
best constant $C$ in%
\begin{equation}
\left\Vert \left(  \sum_{i=1}^{\infty}\left\vert a_{i}\frac{\left\vert
I_{i}^{\ast}\right\vert _{\sigma}}{\left\vert I_{i}^{\ast}\right\vert
^{1-\frac{\lambda}{n}}}\right\vert ^{2}\mathbf{1}_{I_{i}}\right)  ^{\frac
{1}{2}}\right\Vert _{L^{p}\left(  \omega\right)  }\leq C \left(  \sigma,\omega\right)  \left\Vert \left(
\sum_{i=1}^{\infty}\left\vert a_{i}\right\vert ^{2}\mathbf{1}_{I_{i}^{\ast}%
}\right)  ^{\frac{1}{2}}\right\Vert _{L^{p}\left(  \sigma\right)
},\label{quad A2 tailless}%
\end{equation}
where $I_{i}^{\ast}$ is taken over the finitely many cubes $I_{i}^{\ast}$ with
same sidelength as $I_{i}$, which are dyadic relative to $I_{i}$ (meaning
$I_{i}^{\ast}$ and $I_{i}$ must both belong to some grid $\mathcal{D}$),
disjoint from $I_{i}$ but whose closures intersect the closure of $I_{i}$, and
satisfying $\operatorname{dist}\left(  I_{i}^{\ast},I_{i}\right)  \leq
10\ell\left(  I_{i}\right)  $, and all sequences numbers $a_{i}$. There is
also a dual characteristic $A_{p^{\prime}}^{\lambda,\ell^{2}%
,\operatorname*{offset}}\left(  \omega,\sigma\right)  $ defined by
interchanging $\sigma$ and $\omega$, and replacing $p$ by $p^{\prime}$. We
note that when $\sigma$ and $\omega$ are doubling measures, then by taking
$a_{i}$ to be nonzero for only one index $i$, we get $A_{p}^{\lambda}%
(\sigma,\omega)\leq A_{p}^{\lambda,\ell^{2},\operatorname*{offset}}\left(
\sigma,\omega\right)  $ \cite{SaWi}, where
\[
A_{p}^{\lambda}\left(  \sigma,\omega\right)  \equiv\sup_{I\in\mathcal{P}^{n}%
}\frac{\left\vert I\right\vert _{\omega}^{\frac{1}{p}}\left\vert I\right\vert
_{\sigma}^{\frac{1}{p^{\prime}}}}{\left\vert I\right\vert ^{1-\frac{\lambda
}{n}}}\,.
\]


Next we have the scalar Haar testing analogue of Theorem \ref{thm:main_L2} for
$L^{p}$, but because $p$ may not equal $2$ we must assume that $A_{p}%
^{\lambda,\ell^{2},\operatorname*{offset}}\left(  \sigma,\omega\right)
<\infty$ as an additional characterizing condition.

\begin{theorem}
\label{thm:Lp_Haar_Ap_quad} \label{main' p}Suppose $1<p<\infty$, $0\leq
\lambda<n<\infty$. If $\sigma$ and $\omega$ are doubling measures on
$\mathbb{R}^{n}$, $\mathcal{D}$ is a dyadic grid on $\mathbb{R}^{n}$, and
$T^{\lambda}$ is a smooth gradient-elliptic Calder\'{o}n-Zygmund operator on
$\mathbb{R}^{n}$, then%
\[
\mathfrak{N}_{T^{\lambda},p}\left(  \sigma,\omega\right)  \approx
\mathfrak{H}_{T^{\lambda},p}^{\mathcal{D},\operatorname*{glob}}\left(
\sigma,\omega\right)  +\mathfrak{H}_{T^{\lambda,\ast},p^{\prime}}%
^{\mathcal{D},\operatorname*{glob}}\left(  \omega,\sigma\right)
+A_{p}^{\lambda,\ell^{2},\operatorname*{offset}}\left(  \sigma,\omega\right)
+A_{p^{\prime}}^{\lambda,\ell^{2},\operatorname*{offset}}\left(  \omega
,\sigma\right)  \ .
\]

\end{theorem}

\begin{proof}
[Proof of Theorem \ref{main' p}]The proof of this theorem is similar to that
of Theorem \ref{main'}. More precisely, from the slight improvement
\cite[Theorem 1]{AlLuSaUr3} of the main result of \cite{SaWi}, we have the
equivalence
\begin{equation}
\mathfrak{N}_{T^{\lambda},p}\left(  \sigma,\omega\right)  \approx
\mathfrak{T}_{T^{\lambda},p}^{\operatorname{loc}}\left(  \sigma,\omega\right)
+\mathfrak{T}_{T^{\lambda,\ast},p^{\prime}}^{\operatorname{loc}}\left(
\omega,\sigma\right)  +A_{p}^{\lambda,\ell^{2},\operatorname*{offset}}\left(
\sigma,\omega\right)  +A_{p^{\prime}}^{\lambda,\ell^{2},\operatorname*{offset}%
}\left(  \omega,\sigma\right)  .\label{SawWi}%
\end{equation}
Now the proof of Proposition \ref{single} is easily adapted to $1<p<\infty$,
because the argument only involves estimating halos, i.e., thickened boundaries of cubes, and the result of
combining this with (\ref{SawWi}) and the fact that $A_{p}^{\lambda}%
(\sigma,\omega)\lesssim A_{p}^{\lambda,\ell^{2},\operatorname*{offset}}%
(\sigma,\omega)$ when $\sigma,\omega$ are both doubling is that for any fixed
dyadic grid $\mathcal{D}$,
	\begin{equation}\label{eq:p_dyadic_T1}
\mathfrak{N}_{T^{\lambda},p}\left(  \sigma,\omega\right)  \leq C_{\lambda
,n}\left(  \mathfrak{T}_{T^{\lambda},p}^{\mathcal{D},\operatorname*{trip}%
}\left(  \sigma,\omega\right)  +\mathfrak{T}_{\left(  T^{\lambda}\right)
^{\ast},p^{\prime}}^{\mathcal{D},\operatorname*{trip}}\left(  \omega
,\sigma\right)  +A_{p}^{\lambda,\ell^{2},\operatorname*{offset}}\left(
\sigma,\omega\right)  +A_{p^{\prime}}^{\lambda,\ell^{2},\operatorname*{offset}%
}\left(  \omega,\sigma\right)  \right)  .
	\end{equation}
Next we note that the proof of Theorem \ref{Haar} easily extends from $2$ to
$p$ to yield in particular that
	\begin{equation}\label{eq:Haar_p_interval_less_haar}
\mathfrak{T}_{T^{\lambda},p}^{\mathcal{D},\operatorname*{trip}}\left(
\sigma,\omega\right)  \lesssim\mathfrak{H}_{T^{\lambda},p}^{\mathcal{D}%
,\operatorname*{glob}}\left(  \sigma,\omega\right)  :
	\end{equation}
indeed, (\ref{eq:init_nondegen}) doesn't change. And the $L^{p}$ analogue of
(\ref{eq:phi_Haar_testing}), i.e.
	\[
		\left \|  T^{\lambda } _{\sigma}  \frac{\varphi}{\left \|  \varphi \right \|_{L^p (\sigma)}} \right \|_{L^p (\omega)} \lesssim \mathfrak{H}_{T^{\lambda},p}^{\mathcal{D}%
,\operatorname*{glob}}\left(  \sigma,\omega\right) \, , 
	\]follows by estimating the inner-product arising in (\ref{eq:phi_Haar_testing}) using H\"older's inequality with exponents $\frac{1}{p}$ and $\frac{1}{p'}$, and applying \cite[(4.7)]{Saw6},
which implies that if $\sigma$ is doubling, then for all Haar
wavelets $h_{I}^{\sigma}$ on $I$ we have
	\begin{equation}\label{eq:Lp_to_L2_Haar}
	\left ( \frac{1}{\left | I \right|_{\sigma}} \int  \left |  h_{I}^{\sigma}\right |^p d \sigma \right )^{\frac{1}{p}}  \approx\left ( \frac{1}{\left | I \right|_{\sigma}} \int  \left |  h_{I}^{\sigma}\right |^2 d \sigma \right )^{\frac{1}{2}} \, , \text{ and }\left ( \frac{1}{\left | I \right|_{\sigma}} \int  \left |  h_{I}^{\sigma}\right |^{p'} d \sigma \right )^{\frac{1}{p'}}  \approx\left ( \frac{1}{\left | I \right|_{\sigma}} \int  \left |  h_{I}^{\sigma}\right |^2 d \sigma \right )^{\frac{1}{2}}  \,,
	\end{equation}
where the implied constant depends on $p$ and the doubling constant for
$\sigma$. And as for (\ref{eq:init_interval_to_Haar_bd}), replace it 
by
\begin{align}
\int_{3L}\left\vert T_{\sigma}^{\lambda}\frac{\mathbf{1}_{L}}{\left\vert
L\right\vert _{\sigma}^{\frac{1}{p}}}\right\vert ^{p}d\omega=\left\vert
L\right\vert _{\sigma}^{p-1}\int_{3L}\left\vert T_{\sigma}^{\lambda}%
\frac{\mathbf{1}_{L}}{\left\vert L\right\vert _{\sigma}}\right\vert
	^{p}d\omega  &\lesssim\left\vert L\right\vert _{\sigma}^{p-1}\int_{3L}\left\vert
T_{\sigma}^{\lambda}\varphi\right\vert ^{p}d\omega+\left\vert L\right\vert
_{\sigma}^{p-1}\left\vert \int_{3L}\left\vert T_{\sigma}^{\lambda}%
\frac{\mathbf{1}_{K}}{\left\vert K\right\vert _{\sigma}}\right\vert
^{p}d\omega\right\vert \label{eq:p_interval_to_Haar_bd} \\
&  \lesssim\mathfrak{H}_{T^{\lambda}}^{\mathcal{D},\operatorname*{glob}%
}\left(  \sigma,\omega\right)  ^{p}\left\vert L\right\vert _{\sigma}^{p-1}%
\int_{3L}\left\vert \varphi\right\vert ^{p}d\sigma+A_{p}^{\lambda}\left(
\sigma,\omega\right)  ^{p}\nonumber\\
&  \lesssim\mathfrak{H}_{T^{\lambda}}^{\mathcal{D},\operatorname*{glob}%
}\left(  \sigma,\omega\right)  ^{p}+A_{p}^{\lambda}\left(  \sigma
,\omega\right)  ^{p}\,.\nonumber
\end{align}


Finally, combining (\ref{eq:p_dyadic_T1}) and
(\ref{eq:Haar_p_interval_less_haar}) proves Theorem \ref{main' p}.
\end{proof}

In order to remove the quadratic offset conditions $A_{p}^{\lambda,\ell
^{2},\operatorname*{offset}}\left(  \sigma,\omega\right)  $ and $A_{p^{\prime
}}^{\lambda,\ell^{2},\operatorname*{offset}}\left(  \omega,\sigma\right)  $
from Theorem \ref{main' p}, we need to replace the
scalar Haar testing charactristics $\mathfrak{H}_{T^{\lambda},p}%
^{\mathcal{D},\operatorname*{glob}}\left(  \sigma,\omega\right)  $ and
$\mathfrak{H}_{T^{\lambda,\ast},p^{\prime}}^{\mathcal{D},\operatorname*{glob}%
}\left(  \omega,\sigma\right)  $ with the larger \emph{quadratic} Haar testing
characteristics,%
\begin{align*}
\mathfrak{H}_{T^{\lambda},p}^{\ell^{2};\operatorname*{glob}}\left(
\sigma,\omega\right)   &  \equiv\sup_{\left\{  I_{j}\right\}  _{j}\text{ with
}I_{j}\in\mathcal{P}^{n}}\sup\limits_{\left\{  h_{I_{j}}^{\sigma}\right\}
_{j}}\sup_{\left\{  a_{j}\right\}  _{j}\text{ with }a_{j}\in\mathbb{R}}%
\frac{\left\Vert \left(  \sum_{j=1}^{\infty}\left\vert a_{j}T_{\sigma
}^{\lambda}h_{I_{j}}^{\sigma}\right\vert ^{2}\right)  ^{\frac{1}{2}%
}\right\Vert _{L^{p}\left(  \omega\right)  }}{\left\Vert \left(  \sum
_{j=1}^{\infty}\left\vert a_{j}h_{I_{j}}^{\sigma}\right\vert ^{2}\right)
^{\frac{1}{2}}\right\Vert _{L^{p}\left(  \sigma\right)  }},\\
\mathfrak{H}_{T^{\lambda,\ast},p^{\prime}}^{\mathcal{D},\operatorname*{glob}%
}\left(  \omega,\sigma\right)   &  \equiv\sup_{\left\{  I_{j}\right\}
_{j}\text{ with }I_{j}\in\mathcal{P}^{n}}\sup\limits_{\left\{  h_{I_{j}%
}^{\omega}\right\}  _{j}}\sup_{\left\{  a_{j}\right\}  _{j}\text{ with }%
a_{j}\in\mathbb{R}}\frac{\left\Vert \left(  \sum_{j=1}^{\infty}\left\vert
a_{j}T_{\omega}^{\lambda,\ast}h_{I_{j}}^{\omega}\right\vert ^{2}\right)
^{\frac{1}{2}}\right\Vert _{L^{p^{\prime}}\left(  \sigma\right)  }}{\left\Vert
\left(  \sum_{j=1}^{\infty}\left\vert a_{j}h_{I_{j}}^{\omega}\right\vert
^{2}\right)  ^{\frac{1}{2}}\right\Vert _{L^{p^{\prime}}\left(  \omega\right)
}}\,,
\end{align*}
where the supremum in  $\sup\limits_{\left\{  h_{I_{j}}^{\omega}\right\}
_{j}}$ taken over all possible sequences $\left\{  h_{I_{j}}^{\omega}\right\}
_{j}$ of Haar wavelets $h_{I_{j}}^{\omega}$ on intervals $I_{j}$.
Clearly, $\mathfrak{H}_{T^{\lambda},p}^{\operatorname*{glob}}\left(
\sigma,\omega\right)  \leq\mathfrak{H}_{T^{\lambda},p}^{\ell^{2}%
;\operatorname*{glob}}\left(  \sigma,\omega\right)  $.

\begin{theorem}
\label{theorem:Lp_Haar_quad} \label{main' p glob}Let $1<p<\infty$,
$0\leq\lambda<n<\infty$, and $\kappa\in\mathbb{N}$. If $\sigma$ and $\omega$
are doubling measures on $\mathbb{R}^{n}$, $\mathcal{D}$ is a \emph{fixed}
dyadic grid on $\mathbb{R}^{n}$, and $T^{\lambda}$ is a smooth $\kappa
$-elliptic Calder\'{o}n-Zygmund operator on $\mathbb{R}^{n}$, then%
\[
\mathfrak{N}_{T^{\lambda},p}\left(  \sigma,\omega\right)  \approx
\mathfrak{H}_{T^{\lambda},p}^{\ell^{2};\mathcal{D},\operatorname*{glob}%
}\left(  \sigma,\omega\right)  +\mathfrak{H}_{T^{\lambda,\ast},p}^{\ell
^{2};\mathcal{D},\operatorname*{glob}}\left(  \omega,\sigma\right)  \ .
\]

\end{theorem}

\begin{proof}
By Theorem \ref{thm:Lp_Haar_Ap_quad}, and the fact that $\mathfrak{H}%
_{T^{\lambda},p}^{\operatorname*{glob}}\left(  \sigma,\omega\right)
\leq\mathfrak{H}_{T^{\lambda},p}^{\ell^{2};\operatorname*{glob}}\left(
\sigma,\omega\right)  $, it suffices to show
\[
A_{p}^{\lambda,\ell^{2},\operatorname{offset}}\left(  \sigma,\omega\right)
\lesssim\mathfrak{H}_{T^{\lambda},p}^{\ell^{2};\mathcal{D}%
,\operatorname*{glob}}\left(  \sigma,\omega\right)  .
\]
Let%
\[
A_{p}^{\lambda,\ell^{2}}\left(  \sigma,\omega\right)  \equiv\sup_{\left\{
b_{i}\right\}  ,\left\{  I_{i}\right\}  ,\left\{  J_{i}\right\}  }%
\frac{\left\Vert \left(  \sum_{i=1}^{\infty}\left\vert b_{i}\left(
\frac{\left\vert I_{i}\right\vert _{\sigma}}{\left\vert I_{i}\right\vert
^{1-\frac{\lambda}{n}}}\right)  \mathbf{1}_{J_{i}}\right\vert ^{2}\right)
^{\frac{1}{2}}\right\Vert _{L^{p}\left(  \omega\right)  }}{\left\Vert \left(
\sum_{i=1}^{\infty}\left\vert b_{i}\mathbf{1}_{I_{i}}\right\vert ^{2}\right)
^{\frac{1}{2}}\right\Vert _{L^{p}\left(  \sigma\right)  }},
\]
where $J_{i}$ is a dyadic subcube of $I_{i}$ such that $J_{i}$ and $I_{i}$ are
separated by a bounded number of generations. For convenience we write
$E_{I_{i}}^{\lambda}\sigma\equiv\frac{\left\vert I_{i}\right\vert _{\sigma}%
}{\left\vert I_{i}\right\vert ^{1-\frac{\lambda}{n}}}$. Using the
Fefferman-Stein inequality, one may show as in \cite[page 4]{SaWi} or
\cite[page 8]{AlLuSaUr3} that for doubling measures, we have $A_{p}%
^{\lambda,\ell^{2}}\left(  \sigma,\omega\right)  \approx A_{p}^{\lambda
,\ell^{2},\operatorname{offset}}\left(  \sigma,\omega\right)  $. Thus it
suffices to show that
\[
A_{p}^{\lambda,\ell^{2}}\left(  \sigma,\omega\right)  \lesssim\mathfrak{H}%
_{T^{\lambda},p}^{\ell^{2};\mathcal{D},\operatorname*{glob}}\left(
\sigma,\omega\right)  ,
\]
i.e.%

\begin{equation}
\left\Vert \left(  \sum_{i=1}^{\infty}\left\vert b_{i}\left(  E_{I_{i}%
}^{\lambda}\sigma\right)  \mathbf{1}_{J_{i}}\right\vert ^{2}\right)
^{\frac{1}{2}}\right\Vert _{L^{p}\left(  \omega\right)  }\lesssim
\mathfrak{H}_{T^{\lambda},p}^{\ell^{2};\mathcal{D},\operatorname*{glob}%
}\left(  \sigma,\omega\right)  \left\Vert \left(  \sum_{i=1}^{\infty
}\left\vert b_{i}\mathbf{1}_{I_{i}}\right\vert ^{2}\right)  ^{\frac{1}{2}%
}\right\Vert _{L^{p}\left(  \sigma\right)  }.\label{control}%
\end{equation}
Set $a_{i}\equiv b_{i}\sqrt{\left\vert J_{i}\right\vert _{\sigma}}$. Then%
\begin{align*}
&  \left\Vert \left(  \sum_{i=1}^{\infty}\left\vert b_{i}\left(  E_{I_{i}%
}^{\lambda}\sigma\right)  \mathbf{1}_{J_{i}}\right\vert ^{2}\right)
^{\frac{1}{2}}\right\Vert _{L^{p}\left(  \omega\right)  }=\left\Vert \left(
\sum_{i=1}^{\infty}a_{i}^{2}\frac{\left\vert I_{i}\right\vert _{\sigma}^{2}%
}{\left\vert J_{i}\right\vert _{\sigma}\left\vert I_{i}\right\vert ^{2\left(
1-\frac{\lambda}{n}\right)  }}\mathbf{1}_{J_{i}}\right)  ^{\frac{1}{2}%
}\right\Vert _{L^{p}\left(  \omega\right)  }\\
&  \approx\left\Vert \left(  \sum_{i=1}^{\infty}a_{i}^{2}\frac{\left\vert
J_{i}\right\vert _{\sigma}}{\left\vert J_{i}\right\vert ^{2\left(
1-\frac{\lambda}{n}\right)  }}\mathbf{1}_{J_{i}}\right)  ^{\frac{1}{2}%
}\right\Vert _{L^{p}\left(  \omega\right)  }\,.
\end{align*}
Set
\[
\varphi_{i}\equiv\frac{1}{\left\vert L_{i}\right\vert _{\sigma}}%
\mathbf{1}_{L_{i}}-\frac{1}{\left\vert K_{i}\right\vert _{\sigma}}%
\mathbf{1}_{K_{i}},
\]
where we let $I_{i},J_{i},K_{i},L_{i}$ all be related as in the proof of
Theorem \ref{Haar}, with the modification that $K_{i},L_{i}\in\mathfrak{C}%
^{\left(  m\right)  }_{\mathcal{D}}\left(  I_{i}^{\prime}\right)  $ where
$I_{i}^{\prime}$ is a dyadic interval contained in the interval $I_{i}$ having
side length comparable to that of $I_{i}$. Then using the arguments
surrounding (\ref{eq:init_nondegen}) above, we obtain its analogue
\[
\frac{\mathbf{1}_{J_{i}}\left(  x\right)  }{\left\vert J_{i}\right\vert
^{1-\frac{\lambda}{n}}}\lesssim\left\vert T_{\sigma}^{\lambda}\varphi
_{i}\left(  x\right)  \right\vert \mathbf{1}_{J_{i}}\left(  x\right)  ,
\]
and hence
\[
\left\Vert \left(  \sum_{i=1}^{\infty}a_{i}^{2}\frac{\left\vert J_{i}%
\right\vert _{\sigma}}{\left\vert J_{i}\right\vert ^{2\left(  1-\frac{\lambda
}{n}\right)  }}\mathbf{1}_{J_{i}}\right)  ^{\frac{1}{2}}\right\Vert
_{L^{p}\left(  \omega\right)  }\lesssim\left\Vert \left(  \sum_{i=1}^{\infty
}a_{i}^{2}\left\vert J_{i}\right\vert _{\sigma}\left\vert T_{\sigma}^{\lambda
}\varphi_{i}\right\vert ^{2}\mathbf{1}_{J_{i}}\right)  ^{\frac{1}{2}%
}\right\Vert _{L^{p}\left(  \omega\right)  }\,.
\]
Continuing with $\left\Vert \varphi_{i}\right\Vert _{L^{2}\left(
\sigma\right)  }^{2}\approx\frac{1}{\left\vert J_{i}\right\vert _{\sigma}}$ by
doubling, we have%
\[
\left\Vert \left(  \sum_{i=1}^{\infty}a_{i}^{2}\left\vert J_{i}\right\vert
_{\sigma}\left\vert T_{\sigma}^{\lambda}\varphi_{i}\right\vert ^{2}%
\mathbf{1}_{J_{i}}\right)  ^{\frac{1}{2}}\right\Vert _{L^{p}\left(
\omega\right)  }\approx\left\Vert \left(  \sum_{i=1}^{\infty}a_{i}%
^{2}\left\vert T_{\sigma}^{\lambda}\frac{\varphi_{i}}{\left\Vert \varphi
_{i}\right\Vert _{L^{2}\left(  \sigma\right)  }}\right\vert ^{2}%
\mathbf{1}_{J_{i}}\right)  ^{\frac{1}{2}}\right\Vert _{L^{p}\left(
\omega\right)  }.
\]


Given a dyadic cube $I^{\prime}$ and a dyadic cube $K\in\mathfrak{C}%
_{\mathcal{D}}^{\left[  m-1\right]  }\left(  I^{\prime}\right)  $, we can
uniquely determine $K$ by knowing its \textquotedblleft
location\textquotedblright\ $\theta$ within $I^{\prime}$, and then writing
$K=\theta I^{\prime}$. For instance, when $n=1$ we can let $\theta$ denote a
string of $+$'s and $-$'s, where each $\pm$ represents whether an interval is a right or left child of its dyadic parent. The whole string
indicates the ancestral lineage of $K$ relative to $I^{\prime}$. And so now
given $\theta I_{i}^{\prime}$, let $\{\gamma_{j}^{i,\theta}\}_{j=1}^{2^{n}-1}$
be an enumeration of $\Gamma_{\theta I_{i}^{\prime},n}^{\sigma}$, so that we
have the analogue of (\ref{key identity}), namely
\begin{equation}
\frac{\varphi_{i}}{\left\Vert \varphi_{i}\right\Vert _{L^{2}\left(
\sigma\right)  }}=\sum_{\theta}\sum\limits_{j=1}^{2^{n}-1}\left\langle
\frac{\varphi_{i}}{\left\Vert \varphi_{i}\right\Vert _{L^{2}\left(
\sigma\right)  }},h_{\theta I_{i}^{\prime}}^{\sigma,\gamma_{j}^{i,\theta}%
}\right\rangle h_{\theta I_{i}^{\prime}}^{\sigma,\gamma_{j}^{i,\theta}%
}\ .\label{key identity 2}%
\end{equation}


Then by Minkowski's inequality and then quadratic Haar testing we get
\begin{align*}
&  \left\Vert \left(  \sum_{i=1}^{\infty}a_{i}^{2}\left\vert T_{\sigma
}^{\lambda}\frac{\varphi_{i}}{\left\Vert \varphi_{i}\right\Vert _{L^{2}\left(
\sigma\right)  }}\right\vert ^{2}\mathbf{1}_{J_{i}}\right)  ^{\frac{1}{2}%
}\right\Vert _{L^{p}\left(  \omega\right)  }\lesssim\left\Vert \left(
\sum_{i=1}^{\infty}a_{i}^{2}\left\vert T_{\sigma}^{\lambda}\sum_{\theta}%
\sum\limits_{j=1}^{2^{n}-1}\left\langle \frac{\varphi_{i}}{\left\Vert
\varphi_{i}\right\Vert _{L^{2}\left(  \sigma\right)  }},h_{\theta
I_{i}^{\prime}}^{\sigma,\gamma_{j}^{i,\theta}}\right\rangle h_{\theta
I_{i}^{\prime}}^{\sigma,\gamma_{j}^{i,\theta}}\right\vert ^{2}\mathbf{1}%
_{J_{i}}\right)  ^{\frac{1}{2}}\right\Vert _{L^{p}\left(  \omega\right)  }\\
&  \lesssim\sum_{\theta}\sum\limits_{j=1}^{2^{n}-1}\left\Vert \left(
\sum_{i=1}^{\infty}a_{i}^{2}\left\vert T_{\sigma}^{\lambda}\left\langle
\frac{\varphi_{i}}{\left\Vert \varphi_{i}\right\Vert _{L^{2}\left(
\sigma\right)  }},h_{\theta I_{i}^{\prime}}^{\sigma,\gamma_{j}^{i,\theta}%
}\right\rangle h_{\theta I_{i}^{\prime}}^{\sigma,\gamma_{j}^{i,\theta}%
}\right\vert ^{2}\mathbf{1}_{J_{i}}\right)  ^{\frac{1}{2}}\right\Vert
_{L^{p}\left(  \omega\right)  }\\
&  \lesssim\sum_{\theta}\sum\limits_{j=1}^{2^{n}-1}\mathfrak{H}_{T^{\lambda
},p}^{\ell^{2};\mathcal{D},\operatorname*{glob}}\left(  \sigma,\omega\right)
\left\Vert \left(  \sum_{i=1}^{\infty}\left\vert a_{i}\left\langle
\frac{\varphi_{i}}{\left\Vert \varphi_{i}\right\Vert _{L^{2}\left(
\sigma\right)  }},h_{\theta I_{i}^{\prime}}^{\sigma,\gamma_{j}^{i,\theta}%
}\right\rangle h_{\theta I_{i}^{\prime}}^{\sigma,\gamma_{j}^{i,\theta}%
}\right\vert ^{2}\right)  ^{\frac{1}{2}}\right\Vert _{L^{p}\left(
\sigma\right)  }.
\end{align*}
Altogether, we then get%
\begin{align*}
&  \left\Vert \left(  \sum_{i=1}^{\infty}\left\vert b_{i}\left(  E_{I_{i}%
}^{\lambda}\sigma\right)  \mathbf{1}_{J_{i}}\right\vert ^{2}\right)
^{\frac{1}{2}}\right\Vert _{L^{p}\left(  \omega\right)  }\lesssim\sum_{\theta
}\sum\limits_{j=1}^{2^{n}-1}\mathfrak{H}_{T^{\lambda},p}^{\ell^{2}%
;\mathcal{D},\operatorname*{glob}}\left(  \sigma,\omega\right)  \left\Vert
\left(  \sum_{i=1}^{\infty}\left\vert a_{i}\left\langle \frac{\varphi_{i}%
}{\left\Vert \varphi_{i}\right\Vert _{L^{2}\left(  \sigma\right)  }},h_{\theta
I_{i}^{\prime}}^{\sigma,\gamma_{j}^{i,\theta}}\right\rangle h_{\theta
I_{i}^{\prime}}^{\sigma,\gamma_{j}^{i,\theta}}\right\vert ^{2}\right)
^{\frac{1}{2}}\right\Vert _{L^{p}\left(  \sigma\right)  }\\
&  \lesssim\sum_{\theta}\sum\limits_{j=1}^{2^{n}-1}\mathfrak{H}_{T^{\lambda
},p}^{\ell^{2};\mathcal{D},\operatorname*{glob}}\left(  \sigma,\omega\right)
\left\Vert \left(  \sum_{i=1}^{\infty}\left\vert a_{i}h_{\theta I_{i}^{\prime
}}^{\sigma,\gamma_{j}^{i,\theta}}\right\vert ^{2}\right)  ^{\frac{1}{2}%
}\right\Vert _{L^{p}\left(  \sigma\right)  }\,.
\end{align*}
Because $\sigma$ is doubling, we can apply \cite[(4.7)]{Saw6}, namely
\[
\left\Vert h_{\theta I_{i}^{\prime}}^{\sigma,\gamma_{j}^{i,\theta}}\right\Vert
_{L^{\infty}(\sigma)}\lesssim\frac{1}{\sqrt{\left\vert I_{i}\right\vert
_{\sigma}}}\left\Vert h_{\theta I_{i}^{\prime}}^{\sigma,\gamma_{j}^{i,\theta}%
}\right\Vert _{L^{2}(\sigma)}\,,
\]
to get the above is at most a constant times
\[
\mathfrak{H}_{T^{\lambda},p}^{\ell^{2};\mathcal{D},\operatorname*{glob}%
}\left(  \sigma,\omega\right)  \left\Vert \left(  \sum_{i=1}^{\infty}a_{i}%
^{2}\frac{1}{\left\vert I_{i}\right\vert _{\sigma}}\mathbf{1}_{I_{i}}\right)
^{\frac{1}{2}}\right\Vert _{L^{p}\left(  \sigma\right)  }=\mathfrak{H}%
_{T^{\lambda},p}^{\ell^{2};\mathcal{D},\operatorname*{glob}}\left(
\sigma,\omega\right)  \left\Vert \left(  \sum_{i=1}^{\infty}b_{i}%
^{2}\mathbf{1}_{I_{i}}\right)  ^{\frac{1}{2}}\right\Vert _{L^{p}\left(
\sigma\right)  },
\]
which then yields (\ref{control}).
\end{proof}


\section{Alpert wavelets}

\label{section:Alpert}

In this section we consider a much wider
class of bases and frames than the class of weighted Haar wavelets, namely the
weighted \emph{Alpert} wavelets. There is however a price to pay, namely that
the family of truncations of Calder\'{o}n-Zygmund opertors must be further
restricted.

We first recall the construction of weighted Alpert wavelets in
\cite{RaSaWi}\footnote{See \cite{AlSaUr} for the correction of a small
oversight in \cite{RaSaWi}.}. The Alpert wavelets generalize the Haar wavelets
by permitting more vanishing moments, at the price of including polynomials in
the restrictions of the wavelets to children of cubes. Let $\mu$ be a locally
finite positive Borel measure on $\mathbb{R}^{n}$, and fix $\kappa
\in\mathbb{N}$. For each cube $Q$, denote by $L_{Q;\kappa}^{2}\left(
\mu\right)  $ the finite dimensional subspace of $L^{2}\left(  \mu\right)  $
that consists of linear combinations of the indicators of\ the dyadic children
$\mathfrak{C}\left(  Q\right)  $ of $Q$ multiplied by polynomials of degree
less than $\kappa$, and such that the linear combinations have vanishing $\mu
$-moments on the cube $Q$ up to order $\kappa-1$:%
\[
L_{Q;\kappa}^{2}\left(  \mu\right)  \equiv\left\{  f=%
%TCIMACRO{\dsum \limits_{Q^{\prime}\in\mathfrak{C}\left(  Q\right)  }}%
%BeginExpansion
{\displaystyle\sum\limits_{Q^{\prime}\in\mathfrak{C}\left(  Q\right)  }}
%EndExpansion
\mathbf{1}_{Q^{\prime}}p_{Q^{\prime};\kappa}\left(  x\right)  :\int
_{Q}f\left(  x\right)  x^{\beta}d\mu\left(  x\right)  =0,\ \ \ \text{for
}0\leq\left\vert \beta\right\vert <\kappa\right\}  ,
\]
where $p_{Q^{\prime};\kappa}\left(  x\right)  =\sum_{\beta\in\mathbb{Z}%
_{+}^{n}:\left\vert \beta\right\vert \leq\kappa-1\ }a_{Q^{\prime};\beta
}x^{\beta}$ is a polynomial in $\mathbb{R}^{n}$ of degree less than $\kappa$.
Here $x^{\beta}=x_{1}^{\beta_{1}}x_{2}^{\beta_{2}}...x_{n}^{\beta_{n}}$. Let
$d_{Q;\kappa}\equiv\dim L_{Q;\kappa}^{2}\left(  \mu\right)  $ be the dimension
of the finite dimensional linear space $L_{Q;\kappa}^{2}\left(  \mu\right)  $.

For $Q$, let
$\bigtriangleup_{Q;\kappa}^{\mu}$ denote orthogonal projection onto the finite
dimensional subspace $L_{Q;\kappa}^{2}\left(  \mu\right)  $, and let
$\mathbb{E}_{Q;\kappa}^{\mu}$ denote orthogonal projection onto the finite
dimensional subspace%
\[
\mathcal{P}_{Q;\kappa}^{n}\left(  \mu\right)  \equiv
\mathrm{\operatorname*{Span}}\{\mathbf{1}_{Q}x^{\beta}:0\leq\left\vert
\beta\right\vert <\kappa\}.
\]


For a doubling measure $\mu$, it is proved in \cite{RaSaWi} that for any dyadic grid $\mathcal{D}$, we have the
orthonormal decompositions%
\begin{equation}
f=\sum_{Q\in\mathcal{D}}\bigtriangleup_{Q;\kappa}^{\mu}f,\ \ \ \ \ f\in
L_{\mathbb{R}^{n}}^{2}\left(  \mu\right)  ,\ \ \ \ \ \text{where }\left\langle
\bigtriangleup_{P;\kappa}^{\mu}f,\bigtriangleup_{Q;\kappa}^{\mu}f\right\rangle
=0\text{ for }P\neq Q, \label{Alpert expan}%
\end{equation}
where convergence holds both in $L_{\mathbb{R}^{n}}^{2}\left(  \mu\right)  $
norm and pointwise $\mu$-almost everywhere, the telescoping identities%
\begin{equation}
\mathbf{1}_{Q}\sum_{I:\ Q\subsetneqq I\subset P}\bigtriangleup_{I;\kappa}%
^{\mu}=\mathbb{E}_{Q;\kappa}^{\mu}-\mathbf{1}_{Q}\mathbb{E}_{P;\kappa}^{\mu
}\ \text{ \ for }P,Q\in\mathcal{D}\text{ with }Q\subsetneqq P,
\label{telescoping}%
\end{equation}
and the moment vanishing conditions%
\begin{equation}
\int_{\mathbb{R}^{n}}\bigtriangleup_{Q;\kappa}^{\mu}f\left(  x\right)
\ x^{\beta}d\mu\left(  x\right)  =0,\ \ \ \text{for }Q\in\mathcal{D},\text{
}\beta\in\mathbb{Z}_{+}^{n},\ 0\leq\left\vert \beta\right\vert <\kappa\ .
\label{mom con}%
\end{equation}


We have the following bound for the Alpert projections $\mathbb{E}_{I;\kappa
}^{\mu}$ (\cite[see (4.7) on page 14]{Saw6}):
\begin{equation}
\left\Vert \mathbb{E}_{I;\kappa}^{\mu}f\right\Vert _{L_{I}^{\infty}\left(
\mu\right)  }\lesssim E_{I}^{\mu}\left\vert f\right\vert \leq\sqrt{\frac
{1}{\left\vert I\right\vert _{\mu}}\int_{I}\left\vert f\right\vert ^{2}d\mu
},\ \ \ \ \ \text{for all }f\in L_{\operatorname*{loc}}^{2}\left(  \mu\right)
. \label{analogue}%
\end{equation}
Namely \eqref{analogue} allows us to obtain the analogue of \eqref{eq:Lp_to_L2_Haar} for Alpert wavelets.
In terms of Alpert coefficient vectors $\widehat{f}\left(  I\right)
\equiv\left\{  \left\langle f,h_{I;\kappa}^{\mu,\gamma}\right\rangle \right\}
_{\gamma\in\Gamma_{I,n,\kappa}^{\mu}}$ for a choice of an orthonormal basis
$\left\{  h_{I;\kappa}^{\mu,\gamma}\right\}  _{\gamma\in\Gamma_{I,n,\kappa
}^{\mu}}$ of $L_{I;\kappa}^{2}\left(  \mu\right)  $, where $\Gamma
_{I,n,\kappa}$ is a convenient finite index set of size $d_{Q;\kappa}$, we
thus have%
\begin{equation}
\left\vert \widehat{f}\left(  I\right)  \right\vert =\left\Vert \bigtriangleup
_{I;\kappa}^{\sigma}f\right\Vert _{L^{2}\left(  \sigma\right)  }\leq\left\Vert
\bigtriangleup_{I;\kappa}^{\sigma}f\right\Vert _{L^{\infty}\left(
\sigma\right)  }\sqrt{\left\vert I\right\vert _{\sigma}}\leq C\left\Vert
\bigtriangleup_{I;\kappa}^{\sigma}f\right\Vert _{L^{2}\left(  \sigma\right)
}=C\left\vert \widehat{f}\left(  I\right)  \right\vert . \label{analogue'}%
\end{equation}
For notational convenience, we sometimes denote a choice $\left\{
h_{I;\kappa}^{\mu,\gamma}\right\}  _{\gamma\in\Gamma_{I,n,\kappa}^{\mu}}$ of
$\kappa$-Alpert wavelets by $\mathbf{a}_{\kappa}^{\mu}\equiv\left\{
h_{I;\kappa}^{\mu,\gamma}\right\}  _{\gamma\in\Gamma_{I,n,\kappa}^{\mu}}$. As
for the Haar wavelets, whenever we write $h_{I; \kappa}^{\mu}$, we will mean
an $L^{2} \left( \mu\right) $ normalized function in $L^{2} _{I; \kappa}
\left(  \mu\right) $. So as before, if we do not need to be precise about
which specific Alpert wavelet we're referring to, we'll simply write $h_{I;
\kappa}^{\mu}$ for an individual Alpert wavelet, or $\left\{  h_{I; \kappa
}^{\mu} \right\} $ for the collection of $\mu$-weighted Alpert wavelets
on an interval $I$.

Let $\kappa\in\mathbb{N}$,
and define the global and local $\kappa$\emph{-Alpert} testing characteristics
for $T^{\lambda}$ by%
\begin{align*}
\mathfrak{A}_{T^{\lambda};\kappa}^{\operatorname*{glob}}\left(  \sigma
,\omega\right)   &  \equiv\sup_{I\in\mathcal{P}^{n}} \sup\limits_{\left\{
h_{I; \kappa}^{\sigma} \right\}  }\left\Vert T_{\sigma}^{\lambda}h_{I;\kappa
}^{\sigma,\gamma}\right\Vert _{L^{2}\left(  \omega\right)  }\text{ and
}\mathfrak{A}_{T^{\lambda};\kappa}^{\operatorname*{glob}}\left(  \omega
,\sigma\right)  \equiv\sup_{I\in\mathcal{P}^{n} } \sup\limits_{\left\{  h_{I;
\kappa}^{\omega} \right\}  } \left\Vert T_{\omega}^{\lambda,\ast}h_{I;\kappa
}^{\omega}\right\Vert _{L^{2}\left(  \sigma\right)  },\\
\mathfrak{A}_{T^{\lambda};\kappa}^{\operatorname{loc}}\left(  \sigma
,\omega\right)   &  \equiv\sup_{I\in\mathcal{P}^{n}} \sup\limits_{\left\{
h_{I; \kappa}^{\sigma} \right\}  } \left\Vert \mathbf{1}_{I}T_{\sigma
}^{\lambda}h_{I;\kappa}^{\sigma}\right\Vert _{L^{2}\left(  \omega\right)
}\text{ and }\mathfrak{A}_{T^{\lambda};\kappa}^{\operatorname{loc}}\left(
\omega,\sigma\right)  \equiv\sup_{I\in\mathcal{P}^{n}} \sup\limits_{\left\{
h_{I; \kappa}^{\omega} \right\}  } \left\Vert \mathbf{1}_{I}T_{\omega
}^{\lambda}h_{I;\kappa}^{\omega}\right\Vert _{L^{2}\left(  \sigma\right)  },
\end{align*}
and their quadratic counterparts by%
\begin{align*}
\mathfrak{A}_{T^{\lambda};\kappa,p}^{\ell^{2};\operatorname*{glob}}\left(
\sigma,\omega\right)   &  \equiv\sup_{I_{j}\in\mathcal{P}^{n} } \sup
\limits_{\left\{  h_{I_{j} ; \kappa}^{\sigma} \right\} _{j} } \sup_{a_{j}%
\in\mathbb{R}}\frac{\left\Vert \left(  \sum_{j=1}^{\infty}\left\vert
a_{j}T_{\sigma}^{\lambda}h_{I_{j};\kappa}^{\sigma}\right\vert ^{2}\right)
^{\frac{1}{2}}\right\Vert _{L^{p}\left(  \omega\right)  }}{\left\Vert \left(
\sum_{j=1}^{\infty}\left\vert a_{j}h_{I_{j};\kappa}^{\sigma}\right\vert
^{2}\right)  ^{\frac{1}{2}}\right\Vert _{L^{p}\left(  \sigma\right)  }},\\
\mathfrak{A}_{T^{\lambda,\ast};\kappa,p^{\prime}}^{\ell^{2}%
,\operatorname*{glob}}\left(  \omega,\sigma\right)   &  \equiv\sup_{I_{j}%
\in\mathcal{P}^{n}} \sup\limits_{\left\{  h_{I_{j} ; \kappa}^{\omega} \right\}
_{j} }\sup_{a_{j}\in\mathbb{R}}\frac{\left\Vert \left(  \sum_{j=1}^{\infty
}\left\vert a_{j}T_{\omega}^{\lambda,\ast}h_{I_{j};\kappa}^{\omega}\right\vert
^{2}\right)  ^{\frac{1}{2}}\right\Vert _{L^{p^{\prime}}\left(  \sigma\right)
}}{\left\Vert \left(  \sum_{j=1}^{\infty}\left\vert a_{j}h_{I_{j};\kappa
}^{\omega}\right\vert ^{2}\right)  ^{\frac{1}{2}}\right\Vert _{L^{p^{\prime}%
}\left(  \omega\right)  }}.
\end{align*}
We may also replace $\mathcal{P}^{n}$ in the definitions above by a dyadic
grid $\mathcal{D}$ on $\mathbb{R}^{n}$ to get their $\mathcal{D}$
counterparts. Note that we have replaced $\mathfrak{H}$ = fraktur $H$ by
$\mathfrak{A}$ = fraktur $A$ and added the subscript $;\kappa$ when passing
from Haar to Alpert characteristics; indeed, when $\kappa=1$, we recover the familiar Haar testing characteristics. As was mentioned for Haar wavelets in
Section \ref{subsection:frames_weighted_Lp}, the collection of $\kappa$-Alpert
wavelets $\left\{   h_{Q; \kappa}^{\mu, \gamma}\right\}  _{Q\in\mathcal{D},
\gamma\in\Gamma_{I,n ,\kappa} ^{\sigma}}$ is a frame for $L^{p}\left(
\mu\right)  $, $1<p<\infty$. See \cite{RaSaWi,SaWi}.

Here we extend Theorems \ref{main'}, \ref{main' p} and \ref{main' p glob} to Alpert
orthonormal bases and frames respectively. Because of the additional parameter $\kappa$, then the constants implicit in the symbols $\lesssim, \approx$ and $\gtrsim$ may depend on $\kappa$.
\begin{theorem}
\label{main' Alpert}If $\kappa\in\mathbb{N}$, $\sigma$ and $\omega$ are
doubling measures on $\mathbb{R}$, $\mathcal{D}$ is a dyadic grid on
$\mathbb{R}$, and $T^{\lambda}$ is a smooth $\kappa$-elliptic
Calder\'{o}n-Zygmund operator on $\mathbb{R}$, then%
\[
\mathfrak{N}_{T^{\lambda};\kappa}\left(  \sigma,\omega\right)  \approx
\mathfrak{A}_{T^{\lambda};\kappa}^{\mathcal{D},\operatorname*{glob}}\left(
\sigma,\omega\right)  +\mathfrak{A}_{T^{\lambda,\ast};\kappa}^{\mathcal{D}%
,\operatorname*{glob}}\left(  \omega,\sigma\right)  \ .
\]

\end{theorem}

\begin{theorem}
\label{main' p glob Alpert}Let $\kappa\in\mathbb{N}$ and $1<p<\infty$. If
$\sigma$ and $\omega$ are doubling measures on $\mathbb{R}$, $\mathcal{D}$ is
a dyadic grid on $\mathbb{R}$, and $T^{\lambda}$ is a smooth $\kappa$-elliptic
Calder\'{o}n-Zygmund operator on $\mathbb{R}$, then%
\[
\mathfrak{N}_{T^{\lambda};\kappa,p}\left(  \sigma,\omega\right)
\approx\mathfrak{A}_{T^{\lambda};\kappa,p}^{\ell^{2};\mathcal{D}%
,\operatorname*{glob}}\left(  \sigma,\omega\right)  +\mathfrak{A}%
_{T^{\lambda,\ast};\kappa,p^{\prime}}^{\ell^{2};\mathcal{D}%
,\operatorname*{glob}}\left(  \omega,\sigma\right)  \ .
\]
\end{theorem}

The proofs of both of these theorems are virtually identical to the proofs of Theorems \ref{main'} and \ref{main' p glob},
respectively, save for two important points. Recall that in the proof of
Theorem \ref{main'}, given dyadic cubes $I$ and $J$ of equal side length $\ell\left(  I\right)  $
with $\operatorname*{dist}\left(  I,J\right)  \approx\frac{1}{\delta}%
\ell\left(  I\right)  $, and centers $c_{I}$ and $c_{J}$, such that $J\subset
c_{I}+S\left(  \mathbf{v},\delta\right)  $, we showed there exist two
$m$-grandchildren $K,L\in\mathfrak{C}_{\mathcal{D}}^{\left(  m\right)
}\left(  I\right)  $, where $m \lesssim 1$, such that $\operatorname*{dist}\left(  K,L\right)
\approx
\ell\left(  I\right)  $ and $L\subset c_{K}+S\left(  \mathbf{v},\delta\right)
$. Then the function $\varphi$ defined in (\ref{def phi}) by%
\begin{equation}
\varphi\equiv\frac{1}{\left\vert L\right\vert _{\sigma}}\mathbf{1}_{L}%
-\frac{1}{\left\vert K\right\vert _{\sigma}}\mathbf{1}_{K}\ , \label{def phi'}%
\end{equation}
was shown to satisfy two critical properties.

\begin{enumerate}


\item The nondengeneracy inequality
(\ref{eq:init_nondegen}), i.e.,  $T_{\sigma}^{\lambda}\varphi\left(  x\right)$ is of one sign for $x \in J$ and 
\begin{align}\label{eq:nondegen_original}
\left\vert T_{\sigma}^{\lambda}\varphi\left(  x\right)  \right\vert
\gtrsim\frac{1}{\left|  I\right|  ^{1-\frac{\lambda}{n}} } \quad\text{
when } x \in J,
\end{align}

which followed because the integrand%
\begin{equation}
\left(  K^{\lambda}\left(  x,y\right)  -K^{\lambda}\left(  x,c\right)
\right)  \varphi\left(  y\right)  \label{T phi'}%
\end{equation}
in (\ref{T phi pointwise}) doesn't change sign if $K^{\lambda}$ is gradient elliptic.

\item That $\varphi$ is a linear combination of Haar wavelets supported in $I$ up to depth $m$, i.e., the key identity (\ref{key identity}),
\begin{equation}
\frac{\varphi}{\left\Vert \varphi\right\Vert _{L^{2}\left(  \sigma\right)  }%
}=\sum_{M\in\mathfrak{C}_{\mathcal{D}}^{\left[  m-1\right]  }\left(  I\right)
,\gamma\in\Gamma_{M,n}^{\sigma}}\left\langle \frac{\varphi}{\left\Vert
\varphi\right\Vert _{L^{2}\left(  \sigma\right)  }},h_{M}^{\sigma,\gamma
}\right\rangle h_{M}^{\sigma,\gamma} \, ,
\end{equation}
	which followed because
\begin{enumerate}
\item $\frac{\varphi}{\left\Vert \varphi\right\Vert _{L^{2}\left(
\sigma\right)  }}$ is supported in $I$ with $\sigma$-mean zero, and is
constant on the $m$-grandchildren of $I$ for some $m \lesssim1$, and

\item the vector space of such functions is the linear span of the Haar
wavelets $\left\{  h_{M}^{\sigma,\tau}\right\}  _{M\in\mathfrak{C}%
_{\mathcal{D}}^{\left[  m-1\right]  }\left(  I\right)  }$.
\end{enumerate}
\end{enumerate}

In dealing with $\kappa$-Alpert wavelets instead of Haar wavelets, the
function $\varphi$ must be redefined so that

(\textbf{I}) the nondegeneracy condition (\ref{eq:init_nondegen}) continutes to hold

(\textbf{II}) and $\varphi$ belongs to the linear span of the $\kappa$-Alpert
wavelets $\left\{  h_{K;\kappa}^{\sigma,\gamma}\right\}  _{K\in\mathfrak{C}%
_{\mathcal{D}}^{\left[  m-1\right]  }\left(  I \right)  ,\gamma
\in\Gamma\left(  K\right)  }$.

We focus on showing $\varphi$ is a dyadic step function on $I$ with bounded depth $m$ that satisfies the moment-vanishing conditions. Because $\varphi$ is a dyadic step function, higher order polynomials are not needed. To replace the positivity condition (\ref{T phi'}), we assume in addition that the kernel $K^{\lambda}\left(  x,y\right)  $ is $\kappa
$-elliptic, and use linear algebra to get the appropriate moment estimates.

\subsection{Dimension 1 argument}
Fix a configuration of intervals $I$ and $J$ in $\mathcal{D}$ as above, and by translating $\mathcal{D}$, assume without loss of
generality,
\[
I=\left[  0,\ell\left(  I\right)  \right]  \text{ lies to the left of }J.
\]
Let $m$ be a large constant that will be determined later; $m$ may only depend on $\kappa$, the doubling constants of $\sigma$ and $\omega$, and the Calder\'on-Zygmund data for $K$. To make the computations transparent, in what follows, the implicit constant $\approx, \lesssim$ and $\gtrsim$ will not be allowed to depend on $m$ \emph{until we fix  $m$.} For $h = 2^{-m} \ell \left (I \right )$, define 
\[
	K_j = \left [ \left (2j-1 \right )h, 2 j h \right ] \text{ for } 1 \leq j \leq \kappa +1 \, .
\]
Note $K_1 , \ldots, K_{\kappa+1}$ are dyadic subintervals of $I$, each of length $h$ and separated by $h$ as well.  Define 
\[
	\varphi \equiv \frac{1}{ \left | I \right |_{\sigma}} \sum\limits_{i=1}^{\kappa+1} u_{i} \mathbf{1}_{K_i} \, ,
\]
where we will want to choose
\[
	\mathbf{u} \equiv \begin{bmatrix} u_1  \\ \vdots \\ u_{\kappa+1} \end{bmatrix} 
\] so that $\varphi$ satisifes the moment vanishing conditions up to order $\kappa-1$, i.e.,
\[
	\int y^{\ell} \varphi (y) d \sigma (y)= 0  \text{ for } 0 \leq \ell \leq \kappa-1 \, . 
\]
The moment vanishing condition is equivalent to 
\[
	M \mathbf{u} \in \operatorname{Span} \left \{ \mathbf{e}_{\kappa+1} \right \} \, ,
\]
where we define the moment matrix
\begin{align*}
	M &  =\left[
\begin{array}
[c]{cccc}%
\int_{K_{1}}d\sigma\left(
y\right)   & \int_{K_{2}%
}d\sigma\left(  y\right)   & \cdots & \int_{K_{\kappa+1}}d\sigma\left(  y\right)  \\
\int_{K_{1}}yd\sigma\left(
y\right)   & \int_{K_{2}%
}yd\sigma\left(  y\right)   & \cdots & \int_{K_{\kappa+1}}yd\sigma\left(  y\right)  \\
\vdots & \vdots & \ddots & \vdots\\
\int_{K_{1}}y^{\kappa}%
d\sigma\left(  y\right)   & \int_{K_{2}}y^{\kappa}d\sigma\left(  y\right)   & \cdots & \int_{K_{\kappa+1}}%
y^{\kappa}d\sigma\left(  y\right)
\end{array}
\right] \, . 
\end{align*}
So let $\mathbf{u}$ be a unit vector for which 
\begin{align}\label{eq:Alpert_last_comp}
	M \mathbf{u} = \gamma \mathbf{e}_{\kappa+1} \text{ where } \gamma \in \mathbb{R} \, ,
\end{align}
whose existence is justified as follows: the above condition is equivalent to 
\[
	\widetilde{M} \mathbf{u} = 0 \, ,
\]
where $\widetilde{M}$ is obtained by replacing the last row of $M$ by a row of zeros. Since $\widetilde{M}$ maps $\mathbb{R}^{\kappa+1}$ to a $\kappa$-dimensional space, it has nontrivial kernel which must contain a unit vector $\mathbf{u}$. Then $\varphi$ satisfies the moment vanishing conditions up to order $\kappa-1$, and $\varphi$ has $\kappa$-th moment equal to $\frac{\gamma}{\left | I \right |_{\sigma}}$, i.e.,
\begin{align}\label{eq:phi_nondegen_moment}
	\left | \int y^{\kappa} \phi(y) d \sigma (y) \right | = \left | \gamma \right | \, . 
\end{align}
Let us estimate $\left | \gamma \right |$ from below: by \eqref{eq:Alpert_last_comp} and Cramer's rule we have
\[
	 u_i = \frac{\det M^i}{\det M} \gamma  \, ,
\]
where $M^i$ is the matrix $M$ with $i$th column replaced by the column vector $\mathbf{e}_{\kappa+1}$. Thus for all $i$ we have
\[
	\left | \gamma \right | = |u_i| \frac{ | \det M |}{ | \det M^i |  } \, .
\]

The moment matrix $M$ has determinant%
\begin{align*}
	&  \det M =\sum_{\pi \in\operatorname*{Aut}\left(  \left\{
0,1,....,\kappa\right\}  \right)  }\left(  \operatorname{sgn}\pi\right)
\left(  \int_{K_{1}}y_{1}^{\pi\left(  0\right)  }d\sigma\left(
y_{1}\right)  \right)  \left(  \int_{K_{2}}y_{2}^{\pi\left(  1\right)
}d\sigma\left(  y_{2}\right)  \right)  ...\left(  \int_{K_{\kappa+1}}%
y_{\kappa+1}^{\pi\left(  \kappa\right)  }d\sigma\left(  y_{\kappa
+1}\right)  \right)  \\
&  =  \int_{K_{1}}\int_{K_{2}}...\int_{K_{\kappa+1}}\left\{
\sum_{\pi\in\operatorname*{Aut}\left(  \left\{  0,1....,\kappa\right\}
\right)  }\left(  \operatorname{sgn}\pi\right)  y_{1}^{\pi\left(
1\right)  }y_{2}^{\pi\left(  2\right)  }...y_{\kappa+1}^{\pi\left(
\kappa+1\right)  }\right\}  d\sigma\left(  y_{1}\right)  d\sigma\left(
y_{2}\right)  ...d\sigma\left(  y_{\kappa+1}\right)  \\
&  = \int_{K_{1}}\int_{K_{2}}...\int_{K_{\kappa+1}}\det\left[
\begin{array}
[c]{cccc}%
1 & 1 & \cdots & 1\\
y_{1} & y_{2} & \cdots & y_{\kappa+1}\\
\vdots & \vdots & \ddots & \vdots\\
y_{1}^{\kappa} & y_{2}^{\kappa} & \cdots & y_{\kappa+1}^{\kappa}%
\end{array}
\right]  d\sigma\left(  y_{1}\right)  d\sigma\left(  y_{1}\right)
...d\sigma\left(  y_{\kappa+1}\right)  ,
\end{align*}
which by van der Monde's determinant theorem, equals 
\begin{align*}
&  =  \int_{K_{1}}\int_{K_{2}}...\int_{K_{\kappa+1}%
}\left\{  \prod_{1\leq i<j\leq\kappa+1}\left(  y_{j}-y_{i}\right)  \right\}
d\sigma\left(  y_{1}\right)  d\sigma\left(  y_{2}\right)  ...d\sigma\left(
y_{\kappa+1}\right)  \\
&  \approx   \int_{K_{1}}\int_{K_{2}}...\int_{K_{\kappa+1}%
}\left\{  \prod_{1\leq i<j\leq\kappa+1} h  \right\}
d\sigma\left(  y_{1}\right)  d\sigma\left(  y_{2}\right)  ...d\sigma\left(
	y_{\kappa+1}\right) = h^{\frac{\kappa\left(  \kappa+1\right)}{2}} \left \{ \prod\limits_{j=1}^{\kappa+1} \left | K_j \right |_{\sigma} \right \}   \, ,
\end{align*}
since each $h \leq y_{j}-y_{i}\leq 10 \kappa  h$ for $1\leq i<j\leq\kappa+1$.

Because $\det M^i$ and the determinant of the matrix $M$ with $i$th column and $(\kappa+1)$th row removed are equals in absolute value, then similarly
\[
	\det M^i \approx h^{\frac{\left ( \kappa -1\right )\left(  \kappa\right)}{2}} \left \{ \prod\limits_{j=1 \, , j \neq i}^{\kappa+1} \left | K_j \right |_{\sigma} \right \} \, .
\]


Thus, 
\begin{align}\label{eq:gamma_estimate}
	| \gamma | \approx \left | u_i \right | h^{\kappa} \left | K_i \right |_{\sigma} \text{ for all } 1 \leq i \leq \kappa+1 \, .  
\end{align}
Let $i$ be an index so that $\sum\limits_{j=1}^{\kappa+1} \left | u_j \right | \left | K_j \right |_{\sigma} \approx \left | u_i \right | \left | K_i \right |_{\sigma}$.
Now using Taylor's theorem and then the moment vanishing condition, for $x \in J$ we compute
\begin{align}
	\int K^{\lambda} (x,y) \varphi(y) d \sigma(y) &= \int \left \{ K^{\lambda} (x,0)+ \ldots + \partial_{2} ^{\kappa} K^{\lambda} (x,0)  \frac{y^{\kappa}}{\kappa!} + \partial_{2} ^{\kappa+1} K^{\lambda} (x,\theta y)  \frac{y^{\kappa+1}}{(\kappa+1)!} \right \}  \varphi(y) d \sigma(y) \label{eq:Taylor_Alpert_1d}\\
	&= \partial_{2} ^{\kappa} K^{\lambda} (x,0) \int     \frac{y^{\kappa}}{\kappa!} \phi(y) d \sigma(y) + \int \partial_{2} ^{\kappa+1} K^{\lambda} (x,\theta y)  \frac{y^{\kappa+1}}{(\kappa+1)!} \varphi(y) d \sigma(y) \nonumber \, .
\end{align}
By \eqref{eq:phi_nondegen_moment}, \eqref{eq:gamma_estimate}, $\kappa$-ellipticity and the fact that $x \in J$ implies $|x-0| \approx \ell \left (I \right )$, we have  
\[
	\left |\partial_{2} ^{\kappa} K^{\lambda} (x,0) \int     \frac{y^{\kappa}}{\kappa!} \varphi(y) d \sigma(y) \right |\gtrsim \left ( \frac{h}{\ell \left(I \right )} \right)^{\kappa} \frac{ |u_i| \left | K_i \right |_{\sigma}}{ \ell \left ( I \right )^{1- \lambda} \left | I \right |_{\sigma} } \, .
\]And by the Calder\'on-Zygmund estimates, we have 
\[
	\left | \int \partial_{2} ^{\kappa+1} K^{\lambda} (x,\theta y)  \frac{y^{\kappa+1}}{(\kappa+1)!} \varphi(y) d \sigma(y) \right |\lesssim \left ( \frac{h}{\ell \left(I \right )} \right)^{\kappa+1} \frac{\sum\limits_{j=1}^{\kappa +1} \left | u_j \right | \left | K_j \right |_{\sigma}}{ \left | I \right |_{\sigma} \ell \left ( I \right )} \approx \left ( \frac{h}{\ell \left(I \right )} \right)^{\kappa+1} \frac{\left | u_i \right | \left | K_i \right |_{\sigma}}{ \ell \left ( I \right )^{1- \lambda} \left | I \right |_{\sigma} } \, .
\]


Since $h = 2^{-m} \ell \left (I \right)$, then fixing $m$ to be some sufficiently large constant, depending only on $\kappa$, the doubling constants of $\sigma$ and $\omega$, and the Calder\'on-Zygmund data of $T$, yields the first term on the right of \eqref{eq:Taylor_Alpert_1d} dominates, $\int K^{\lambda} (x,y) \varphi(y) d \sigma(y)$ is of one sign, and 
\[
	\left | \int K^{\lambda} (x,y) \varphi(y) d \sigma(y) \right | \approx \left ( \frac{h}{\ell \left(I \right )} \right)^{\kappa} \frac{ |u_i| \left | K_i \right |_{\sigma}}{ \ell \left ( I \right )^{1- \lambda} \left | I \right |_{\sigma} } \approx \left ( \frac{h}{\ell \left(I \right )} \right)^{\kappa} \frac{1}{\ell \left ( I \right )^{1- \lambda}}  \sum\limits_j |u_j| \frac{\left | K_j \right |_{\sigma}}{ \left | I \right |_{\sigma}} \, .
\]
Since $m$ is fixed, any constant depending on $m$ may be absorbed into the symbols $\lesssim, \approx $ and $\gtrsim$. Since $h = 2^{-m} \ell \left (I \right)$, then 
\[
	\left | \int K ^{\lambda} (x,y) \phi(y) d \sigma(y) \right | \approx \frac{1}{\ell \left (I \right )^{1- \lambda}}   \sum\limits_j |u_j|  \frac{\left | K_j \right |_{\sigma} }{ \left | I \right |_{\sigma} } \, .
\]
By the doubling of $\sigma$, we have $\left | K_j \right |_{\sigma} \approx \left | I \right |_{\sigma}$, which when combined with the fact that $\sum\limits_j |u_j| \approx 1$, yields the nondegeneracy condition \eqref{eq:nondegen_original}.

Finally, because $m \lesssim 1$, then $\varphi$ is a linear combination of indicators of dyadic descendents of $I$, all at most a bounded number of generations below $I$, and $\varphi$ satisfies the moment vanishing conditions. Hence $\varphi$ is a linear combination of a bounded number of $\kappa$-Alpert wavelets.

The proof of Theorem \ref{main' Alpert} is thus completed in the case $n=1$, by
following the appropriate part of the proof of Theorem \ref{main'},\ and by
verifying that the triple Alpert testing characteristics are controlled using
the argument from the proof of Theorem \ref{Haar}, which does \emph{not} use
moment vanishing conditions. Using the proof of Theorem \ref{main' p glob}, the above arguments
can be adapted from orthonormal bases to frames, which proves the case $n=1$
of Theorem \ref{main' p glob Alpert}.

\subsection{The higher dimensional argument}

In extending the proof in dimension $1$ to dimension $n\geq 2$, we encounter two obstacles requiring key changes. One is that we have $\kappa$-ellipcity in only one of the $n$ directions, so we must configure our cubes so that the other directions are negligeable when we do a Taylor expansion of $K$. The other obstacle is the lack of a Vandermonde determinant identity. We get around both obstacles by dilating approrpiately along the coordinate axes, since the determinant and the Taylor expansion both let us factor out the dilation factors, which helps us obtain quantitative estimates. 

Let $I$ and $J$ be cubes in the same configuration as before in $\mathcal{D}$, and assume without loss of generality that the origin is the furthest vertex of $I$ from $J$. We must construct a function $\varphi$ supported on finitely many dyadic subcubes of $I$ which has all moments vanishing up to order $\kappa-1$, i.e.,
\[
	\left \langle \varphi, y^{\alpha} \right \rangle_{\sigma} = \int \varphi (y) y^{\alpha} d \sigma \left ( y \right )  = 0 \text{ for all } \left | \alpha \right |\leq \kappa-1 \, .
\]
This is equivalent to $\varphi$ being in the orthogonal complement of the linear span of the polynomials of degree at most $\kappa -1$. This latter set is invariant under a linear change of variables $x=Ay$, so $\varphi$ having vanishing moments up to order $\kappa-1$ is invariant under any linear change of variables.

Namely, we can change variables by a rotation or a reflection. By first rotating appropriately, assume without loss of generality that $K$ is $\kappa$-elliptic with respect to the vector $\mathbf{v} = \mathbf{e}_1$. The grid $\mathcal{D}$ and the cubes $I$ and $J$ we are considering may no longer be axis-parallel, but this will not matter for us. And by then reflecting, assume without loss of generality that the center of $I$ lies in the first quadrant 
\[
	\left ( \mathbb{R}_+ \right )^n = \left \{ (x_1, \ldots, x_n) \in \mathbb{R}^n ~:~ x_j \geq 0 \right \} \, .
\]
We will eventually stretch $I$ vertically to enclose many points in the first coordinate axis, and having the center of $I$ in the first quadrant will allow us to uniformly estimate the dilating factor needed.

But we primarily do this change of variables to simplify notation with multi-indices. For instance, because of our rotation, note that $\kappa$-ellipticity for the kernel $K$ means  
\[
	\left | \partial^{(\kappa ,0,\ldots, 0)} _{y} K (x,y) \right | \approx \frac{1}{\left | x - y \right|^{n+\kappa}} \, \text{ for all } y \in I, x \in J \, . 
\]
 

 Let us establish some notation. Since we are working on $\mathbb{R}^n$, define the relevant sets of multi-indices 
\[
	\mathcal{E}^* \equiv \{ \beta \in \mathbb{N}^n~:~ |\beta| \leq \kappa-1 \} \text{ and } \mathcal{E} \equiv \mathcal{E} ^* \cup \{ (\kappa, 0,\ldots, 0)\} \, . \] We order both sets of multi-indices by writing $\alpha \prec \beta$ when 
\[
	 |\alpha| < |\beta|
 \]
 or
 \[
	|\alpha| = |\beta| \text{ and } \sum\limits_i  \alpha_i 10^i \leq \sum\limits_i \beta_i 10^i  \, . 
\]
The second condition isn't important: any ordering $\prec$ which respects the first condition will suffice for us. 

Let $N \equiv \left | \mathcal{E} \right |$. Then we imbue $\mathbb{R}^{N}$ with the ordered orthonormal basis $\left ( \mathbf{e}_{\alpha} \right )_{\alpha \in \mathcal{E}}$, where the order is given by $\mathcal{E}$. Given $\alpha \in \mathcal{E}$, we also write  
\[
	\alpha = (\alpha_1, \alpha') \in \mathbb{N} \times \mathbb{N}^{n-1}
\]
where $\alpha_1$ is the first coordinate of $\alpha$. We will need to consider $N \times N$ matrices  
\[
	M = \left ( M_{\alpha, \beta} \right )_{\alpha, \beta \in \mathcal{E}} \, :
\]
we adopt the convention that $\alpha$ and $\beta$ always parametrize the rows and columns of a matrix, respectively.

We will also consider collections of points $(x_{\beta})_{\beta \in \mathcal{E} }$ indexed by the multi-indices $\beta \in \mathbb{N}^n$, i.e., for each $\beta \in \mathbb{N}^n$ we have a point 
\[
	x_{\beta} = \left ( (x_{\beta})_1 ,\ldots , (x_{\beta})_n \right ) \in \mathbb{R}^n \, ,
\] where $\beta \in \mathcal{E}$, for a total of $N = |\mathcal{E}|$ points in $\mathbb{R}^n$. While occasionally we will view the tuple $ (x_{\beta})_{\beta \in \mathcal{E}}$ as a point in $\left ( \mathbb{R}^n \right )^{N}$, we will more often think of $(x_{\beta})_{\beta \in \mathcal{E}}$ as $N$ points all in the \emph{same} Euclidean space $\mathbb{R}^n$.

Let $\delta \in \left (0, 1 \right )$ and $m \in \mathbb{N}$, and set $h = 2^{-m} \ell \left ( I \right)$: the large integer $m$ and the small constant $\delta$ will be chosen later, but will only depend on $\kappa$, the dimension $n$, the doubling constants of $\sigma$ and $\omega$, and the Calder\'on-Zygmund data for $T$. Again for clarity in the proof, until we have fixed $\delta, m$, we will not allow the constants implicit in the symbols $\lesssim , \approx$ or $\gtrsim$ to depend on $\delta$ and $m$.

Define 
\[
	\varphi = \frac{1}{\left | I \right|_{\sigma}} \sum\limits_{\beta \in \mathcal{E}} u_{\beta} \mathbf{1}_{K_{\beta}} \, ,
\]
where the cubes $\{ K_{\beta}\}$ are dyadic subcubes of $I$ of sidelength $\delta^2 h$ which will be chosen later but should be thought of as follows: the cubes $K_{\beta}$ are distance $\lesssim \delta h$ away from the $\mathbf{e}_1$-axis, are distance $\approx h$ from the origin and each other. Also set $\mathbf{u} = \left (u_{\beta} \right )_{\beta \in \mathcal{E}} \in \mathbb{R}^N$.

\begin{figure}[ht]\label{fig:cubes_Alpert_slant}
  \fbox{\includegraphics[width=0.75\linewidth]{cubes_Alpert_slant.png}}
	\caption{The cubes $K_{\beta}$ within $I$. }
\end{figure}

 Then the moment vanishing for $\varphi$ is equivalent to 
\begin{equation}\label{eq:moment_vanishing_general_matrix}
	M \mathbf{u} = \gamma \mathbf{e}_{(\kappa, 0, \ldots, 0)} \, , \quad \gamma \in \mathbb{R} \, ,
\end{equation}
where the moment matrix
\[
	M \equiv \begin{pmatrix} \int_{K_{\beta}} \left ( x_{\beta} \right )^{\alpha} d \sigma (x_{\beta}) \end{pmatrix}_{\alpha, \beta \in \mathcal{E}}  \, .
\]
Let $\mathbf{u}$ be a unit vector satisfying \eqref{eq:moment_vanishing_general_matrix}: such a unit vector exists because if we consider the matrix $\widetilde{M}$ obtained from $M$ by replacing its last row, i.e. the $\alpha$-th row where $\alpha = \kappa \mathbf{e}_1$, by a row of zeros, then $\widetilde{M}$ must have a nontrivial kernel because it maps an $N$-dimensional space into an $(N-1)$-dimensional space, and so we can take $\mathbf{u}$ to be a unit vector in the kernel. 

To estimate the moment of $\varphi$ corresponding to the ellipticity of $K$, we estimate $\gamma$. Cramer's rule applied to \eqref{eq:moment_vanishing_general_matrix} yields
\begin{equation}\label{eq:Cramer_high_dim}
	\left | u_{\beta} \right | = \left | \gamma \right | \frac{\left | \det M^{\beta} \right | }{\left | \det M \right |} \, ,
\end{equation}
where $M^{\beta}$ is simply the matrix $M$ with the $\beta$-th column vector replaced by the column vector $\mathbf{e}_{(\kappa, 0, \ldots, 0)}$.

 Because we lack a Vandermonde determinant formula in higher dimensions, we must estimate $\left | \det M \right |$ differently from before. First note that if we set 
\[
	\bar{x}_{\beta} \equiv \left ( \frac{(x_{\beta})_1}{h}, \frac{ (x_{\beta})_2}{\delta h}, \ldots,  \frac{ (x_{\beta})_n}{\delta h} \right )
\]
and let $H$ denote the diagonal matrix with $(\alpha, \alpha)$-th entry equal to $h^{|\alpha|} \delta^{|\alpha'|}$, 
then 
$M$ factors as 
\[
	M = H M' \, , \quad \text{ where } M' \equiv \begin{pmatrix} \int_{K_{\beta}} \bar{x}^{\alpha} _{\beta} d \sigma (x_{\beta}) \end{pmatrix}_{\alpha, \beta \in \mathcal{E}} \, . 
\]
Then 
\[
	\det H = h ^{\sum\limits_{\alpha \in \mathcal{E}} |\alpha|} \delta ^{\sum\limits_{\alpha \in \mathcal{E}} |\alpha'|} =  h ^{\kappa} \left ( h^{\sum\limits_{\alpha \in \mathcal{E}^*} |\alpha|} \delta ^{\sum\limits_{\alpha \in \mathcal{E}^*} |\alpha'|} \right ) \, .
\]
As for $M'$, we get
\[
	\det M' =  \det \begin{pmatrix} \int_{K_{\beta}}  \bar{x}^{\alpha} _{\beta} d \sigma (x_{\beta}) \end{pmatrix}_{\alpha, \beta \in \mathcal{E}} = \sum\limits_{\pi \in \operatorname{Aut} \left (\mathcal{E} \right )} \operatorname{sgn} (\pi) \prod\limits_{\beta \in \mathcal{E}} \int_{K_{\beta}}  \bar{x}^{\pi(\beta)} _{\beta} d \sigma (x_{\beta}) \, .
\]
Applying Fubini and noting that $\mathbf{x} \equiv \left ( x_{\beta} \right )_{\beta \in \mathcal{E}}$ belongs to the product space $\mathbf{K} \equiv \prod\limits_{\beta \in \mathcal{E}} K_{\beta}$, which we can imbue with the product measure $\boldsymbol{\sigma} \equiv \otimes_{\beta \in \mathcal{E}} \sigma$ we see that 
\begin{equation}\label{eq:det_Fubini}
	\det M' = \int\limits_{\mathbf{K}} \sum\limits_{\pi \in \operatorname{Aut} \left (\mathcal{E} \right )} \operatorname{sgn} (\pi)   \bar{x}^{\pi(\beta)} _{\beta}  d  \boldsymbol{\sigma}  (\mathbf{x})  = \int\limits_{\mathbf{K} } \det M'' (\bar{\mathbf{x}})  d  \boldsymbol{\sigma}  (\mathbf{x}) \, ,
\end{equation}
where
\[
	\bar{\mathbf{x}} \equiv \left ( \bar{x}_{\beta} \right )_{\beta \in \mathcal{E}} \, ,
\]
and where we define the matrix function 
\[
	M'' \left (\left ( z_{\beta}\right )_{\beta \in \mathcal{E}} \right ) \equiv \begin{pmatrix} z_{\beta} ^{\alpha} \end{pmatrix}_{\alpha, \beta \in \mathcal{E}} \, .
\]
In what follows, we view $M''$ as a function of the point $\mathbf{z} \equiv (z_{\beta})_{\beta \in \mathcal{E}} \in \left (\mathbb{R}^{n} \right)^N$.
\begin{lemma}\label{lemma:det_nondegen}
	The function $\mathbf{z} \mapsto \det M'' (\mathbf{z})$ is nonzero on a dense subset of $\mathbb{R}^{Nn}$.	
\end{lemma}
\begin{proof}
	It suffices to show that on each open set $\mathbf{O}$ of the form
	\[
		\mathbf{O} \equiv \prod\limits_{\beta \in \mathcal{E}} \left ( O_{\beta} \right )_{\beta} \subset \mathbb{R}^{Nn} \, ,
	\]
	there exist a point $\mathbf{z} = \left ( z_{\beta} \right )_{\beta \in \mathcal{E}} \in \mathbf{O}$ such that $\det M'' \left (\mathbf{z} \right ) \neq 0$. 

	Assume to the contrary $\det M''$ vanishes everywhere on $\mathbf{O}$. Then all its derivatives should vanish on $\mathbf{O}$. We will show that one of its derivatives is nonzero on $\mathbf{O}$, yielding a contradiction. Let $\alpha^* \equiv \kappa \mathbf{e}_1$ be the last index in $\mathcal{E}$. Then differentiating $\det M''$ with respect to $\partial^{\alpha^*} _{z_{\alpha^*}}$ and using the Laplace expansion of the determinant yields 
	\[
		\partial^{\alpha^*} _{z_{\alpha^*}} \det M'' \left ( \mathbf{z} \right )  =  c_{\alpha^*}   \det \left ( z_{\beta} ^{\alpha} \right )_{\alpha, \beta \in \mathcal{E}^*} \, , 
	\]
	where $c_{\alpha^*}$ is a nonzero constant independent of $\mathbf{z} = \left ( z_{\beta} \right )_{\beta \in \mathcal{E}}$.

	Then it suffices to show $\det \left ( z_{\beta} ^{\alpha} \right )_{\alpha, \beta \in \mathcal{E}^*} \neq 0$ on the open set 
	\[
		\mathbf{O}^* \equiv \prod\limits_{\beta \in \mathcal{E}^*} O_{\beta} \subset \left ( \mathbb{R}^n \right)^{N-1}\, .
	\]
	If $\alpha^{**}$ is the last term in $\mathcal{E}^*$, we get that
	\[
		\partial_{z_{\alpha^{**}}} ^{\alpha^{**}} \det \left ( z_{\beta} ^{\alpha} \right )_{\alpha, \beta \in \mathcal{E}^*} = \pm \det \left ( z_{\beta} ^{\alpha} \right )_{\alpha, \beta \in \mathcal{E}^{**}} \ , 
	\]
	where $\mathcal{E}^{**}$ is missing the last two terms of $\mathcal{E}$. Iterting this process and recognizing the multi-index $\mathbf{0}$ as the first element in $\mathcal{E}$, we see we must show that
	\[
		\det \left ( z_{\beta} ^{\alpha} \right )_{\alpha, \beta \in \{ \mathbf{0} \}} 
	\]
	is nonzero on $O_{\mathbf{0}}$. This in fact equals $1$ since $\left ( z_{\beta} ^{\alpha} \right )_{\alpha, \beta \in \{ \mathbf{0} \}}$ is a $1 \times 1$ matrix with constant entry $1$.	
	\end{proof}

First pick a configuration of points $(z_{\beta})_{\beta \in \mathcal{E}}$ in the first quadrant $\left ( \mathbb{R}_{+} \right )^n$ below any of the diagonal hyperplanes $\{ x_1 = x_j\}$ for $j=2, \ldots , n$: the specific choice configuration will not matter, we only need that the points are all separated by distance $\approx 1$ and are distance $\approx 1$ from the origin. To be concrete, we take $(z_{\beta})_{\beta \in \mathcal{E}}$ to be the first $N$ points in the collection 
\[
	\left \{ \frac{1}{2} \mathbf{e}_2 + t \mathbf{e}_1 \in \mathbb{R}^n  ~:~ t\in \mathbb{Z}_{+} \right \} \, . 
\]
 By Lemma \ref{lemma:det_nondegen}, we may perturb the points $(z_{\beta})_{\beta \in \mathcal{E}}$ slightly, by a distance of at most $\frac{1}{4}$, so that the function
\[
	\left | \det (z_{\beta}^{\alpha} )_{\alpha, \beta \in \mathcal{E}} \right |  \geq c(\kappa) \, ,
\]
where $c(\kappa)$ is some positive constant only depending on $\kappa$.
By continuity, there exists some constant $\epsilon \in (0, \frac{1}{4} )$, only depending on $\kappa$ and $n$, such that whenever $y_{\beta}$ is a point in the ball 
\[
 B_{\beta} \equiv B_{\epsilon } ( z_{\beta}) \, ,
\]
we have
\begin{equation}\label{eq:det_nondegen}
\left | \det (y_{\beta}^{\alpha} )_{\alpha, \beta \in \mathcal{E}} \right| \approx 1 \, ,
\end{equation}
and the function in between absolute values is of one sign. Note the balls $B_{\beta}$ are each separated by distance $\approx 1$, and are distance $\approx 1$ from the origin. These balls exist independently of $I$, and only depend on $K$. 


Define the function 
\[
	\hat{x} \equiv \frac{1}{h} x = \frac{1}{2^{-m} \ell \left ( I \right )} x  \, .
\]
Given $A \subset \mathbb{R}^n$, define 
\[
	\hat{A} \equiv \left \{ \hat{x} ~:~ x \in A \right \} \, . 
\]
Then $\hat{I}$ is a rescaling of the cube $I$ with sidelength $2^{m}$. Because the center $c_I$ of $I$ belongs to the first quadrant $\mathbb{R}_{+} ^n$, then so does the center $c_{\hat{I}}$ of $\hat{I}$. Choose $m$ large enough so that so that $c_{\hat{I}}$ extends past the balls $B_{\beta}$ along the $\mathbf{e}_1$-axis, i.e., 
\[
	\left( c_I \right )_1 \geq \sup\limits_{\beta \in \mathcal{E}}\sup\limits_{y_{\beta} \in B_{\beta}} \left (y_{\beta} \right )_{1} \, . 
\]

By this re-scaling, the dyadic descendents of $\hat{I}$ of generation $m$ are of unit length. For $\delta < \frac{1}{2}$, let $\left \{ \hat{K} \right \}$ denote a covering of the $ \delta$-neighorhood of the $\mathbf{e}_1$ axis within $\hat{I}$ by dyadic subcubes of $\hat{I}$ of sidelength $\delta^2$.

Given $\hat{x} \in \mathbb{R}^n$, define
\[
	\bar{x} \equiv \left ( \hat{x}_1, \frac{\hat{x}_2}{\delta} , \ldots, \frac{\hat{x}_n}{\delta} \right ) \, , 
\]
which coincides with the definition
\[
	\bar{x} = \left ( \frac{x_1}{h}, \frac{x_2}{ \delta h}, \ldots, \frac{x_n}{ \delta h} \right ) \, . 
\]
And again given a set $\hat{A} \subset \mathbb{R}^n$, define
\[
	\bar{A} \equiv \left \{ \bar{x} ~:~ \hat{x} \in A \right \} \, . 
\] 

\begin{figure}[ht]\label{fig:cubes_Alpert_slant_stretch}
  \fbox{\includegraphics[width=0.75\linewidth]{cubes_Alpert_slant_stretch.png}}
	\caption{The stretched cubes $\overline{K}_{\beta}$ within $\overline{I}$. }
\end{figure}

Because of how we chose the points $z_{\beta}$, and because the center of $\hat{I}$ is in the first quadrant, we have that 
\[
	\left ( \bigcup\limits_{\beta} B_{\beta} \right ) \subset \bar{I} \cap \left \{z \in \mathbb{R}^n ~:~ 0 \leq z_1 \leq 1 \right \} \, . 
\]
 The right side has a covering $\{\bar{K}\}$, i.e., the images of the  cover $\{\hat{K}\}$ of the $\delta$-neighborhood of the $\mathbf{e}_1$ axis within $\hat{I}$.

Let $\bar{K}_{\beta}$ denote the element of $\{\bar{K}\}$ containing the point $z_{\beta}$. Note that $\bar{K}_{\beta}$ is an object of dimensions $\approx \delta^2$ in the $\mathbf{e}_1$ direction, and $\approx \delta$ in the directions $\mathbf{e}_2, \ldots, \mathbf{e}_n$. For $\delta$ sufficiently small and only depending on the radius $\epsilon = \epsilon (n, \kappa)$ of the balls $B_{\beta}$, we in fact have that $\bar{K}_{\beta} \subset B_{\beta}$ for each $\beta$. By \eqref{eq:det_nondegen}, if $\bar{x}_{\beta} \in \bar{K}_{\beta}$, then 
\begin{equation}\label{eq:det_nondegen_bars}
	|\det M'' ( (\bar{x}_{\beta} )_{\beta\in \mathcal{E}} ) | \approx 1 \, ,
\end{equation}
and the determinant in between absolute values is of one sign.

Then $\bar{K}_{\beta}$ is the image of some dyadic subcube $K_{\beta}$ of $I$ under the composition of the $\hat{~}$ and  $\bar{~}$ maps. Then for all $x_{\beta} \in K_{\beta}$, we have $\bar{x}_{\beta} \in \bar{K}_{\beta}$ and so \eqref{eq:det_nondegen_bars} still holds, and the determinant is of one sign. Furthermore, each $K_{\beta}$ is of distance $\lesssim h \delta$ from the $\mathbf{e}_1$ axis and distance $\approx h$ from the origin.

Thus for $\mathbf{x} = (x^{\beta})_{\beta} \in \mathbf{K}$, by \eqref{eq:det_Fubini} we have
\[
	\left | \det M' \right | \approx \prod\limits_{\beta} \left | K_{\beta} \right |_{\sigma} \, ,
\]
and hence
\[
	\left | \det M \right | = \left | \det H \right | \left |  \det M' \right | \approx h ^{\kappa} \left ( h^{\sum\limits_{\alpha \in \mathcal{E}^*} |\alpha|} \delta ^{\sum\limits_{\alpha \in \mathcal{E}^*} |\alpha'|} \right ) \prod\limits_{\beta \in \mathcal{E}} \left | K_{\beta } \right |_{\sigma} \, . 
\]
Similarly, by the Laplace expansion of the determinant, we have $\left | \det M^{\beta} \right |$ equals the absolute value of the matrix obtained by deleting the $\beta$-th column and last row of $M$. From this last matrix, we can factor out a diagonal matrix $H$ whose $(\alpha, \alpha)$-th entry equals $h^{|\alpha|} \delta^{|\alpha'|}$, where $\alpha$ varies over $\mathcal{E}^*$ now, and applying the same logic as above we get
\[
	\left | \det M^{\beta} \right | \approx h ^{\sum\limits_{\alpha \in \mathcal{E}^*} |\alpha|} \delta ^{\sum\limits_{\alpha \in \mathcal{E}^*} |\alpha'|} \prod\limits_{\alpha \in \mathcal{E} ~:~ \alpha \neq \beta } \left | K_{\alpha} \right |_{\sigma} \, .
\]
Thus with our initial application of Cramer's rule \eqref{eq:Cramer_high_dim}, we get
\[
	\left | u_{\beta} \right | = \left | \gamma \right | \left | \frac{\det M^{\beta}}{\det M} \right | \approx \left | \gamma \right | \frac{1}{h^{\kappa} \left | K_{\beta}\right |_{\sigma}} \, ,
\]
or rather
\begin{equation}\label{eq:gamma_many_beta}
	\left | \gamma \right | \approx \left | u_{\beta} \right | \left | K_{\beta} \right |_{\sigma} h^{\kappa}  \text{ for all } \beta \in \mathcal{E} \, .
\end{equation}

Now let $\beta^* \in \mathcal{E}$ be a multi-index such that $\sum\limits_{\beta \in \mathcal{E} } \left | u_{\beta} \right | \left | K_{\beta} \right |_{\sigma} \approx \left | u_{\beta^*} \right | \left | K_{\beta^*} \right |_{\sigma}$.

Using Taylor's theorem with $x \in J$, we have $\int K^{\lambda} (x,y) \varphi (y) d\sigma(y)$ equals 
\begin{align*}
	\int \left \{ \partial^{\left ( \kappa, 0, \ldots, 0 \right )} _y K ^{\lambda} (x,0) \frac{y_1^{\kappa}}{\kappa!}  +  \sum\limits_{|\alpha| = \kappa ~:~ \alpha \neq \kappa \mathbf{e}_1 } c_{\alpha} \partial^{\alpha} _y  K^{\lambda} (x,0) y^{\alpha} +  \sum\limits_{|\alpha| = \kappa +1 } c_{\alpha} \partial^{\alpha} _y  K ^{\lambda} (x,\theta y) y^{\alpha}\right \} \varphi (y) d \sigma(y) \, . 
\end{align*}
Using \eqref{eq:gamma_many_beta} and $\kappa$-ellipticity we estimate the first term on the right by 
\[
	\left | \partial^{\left ( \kappa, 0, \ldots, 0 \right )} _y K ^{\lambda} (x,0) \int  \frac{y_1^{\kappa}}{\kappa!} \varphi (y) d \sigma (y) \right |\approx \left | \partial^{\kappa} _1 K ^{\lambda} (x,0) \right | \left | \int   y_1^{\kappa} \varphi(y) \right | \approx  \left ( \frac{h}{\ell \left (I \right)} \right )^{\kappa} \frac{ \left | u_{\beta^*} \right | \left | K_{\beta^*} \right |_{\sigma}}{  \left |I \right |^{1 - \frac{\lambda}{n}} \left | I \right|_{\sigma}}   \, .
\]
 We estimate the second term using the CZ estimates and that each cube $K_{\beta}$ is distance $\leq \delta$ from the $\mathbf{e}_1$-axis: 
\[
	\left |  \sum\limits_{|\alpha| = \kappa ~:~ \alpha \neq \kappa \mathbf{e}_1 } c_{\alpha} \partial^{\alpha} _y  K ^{\lambda} (x,0) \int  y^{\alpha}\varphi (y)  d \sigma(y)     \right |	\lesssim \left ( \frac{h}{\ell \left (I \right)} \right )^{\kappa} \frac{\delta}{ \left |I \right |^{1 - \frac{\lambda}{n}} } \sum\limits_{\beta \in \mathcal{E}} \left | u_{\beta} \right |  \frac{\left | K_{\beta} \right |_{\sigma} }{  \left | I \right|_{\sigma} }   \approx \left ( \frac{h}{\ell \left (I \right)} \right )^{\kappa} \frac{\delta \left | u_{\beta^*} \right | \left | K_{\beta^*} \right |_{\sigma} }{ \left |I \right|^{1 - \frac{\lambda}{n}} \left | I \right |_{\sigma } }   \, .
\]
The third term can also be estimated by the CZ estimates, 
\[
	\left | \int \sum\limits_{|\alpha| = \kappa +1 } c_{\alpha} \partial^{\alpha} _y  K ^{\lambda} (x,\theta y) y^{\alpha} \varphi (y) d \sigma(y) \right | \lesssim \left ( \frac{h}{\ell \left (I \right)} \right )^{\kappa+1}  \frac{1}{ \left |I \right |^{1 - \frac{\lambda}{n}} } \sum\limits_{\beta \in \mathcal{E}} \left | u_{\beta} \right |  \frac{\left | K_{\beta} \right |_{\sigma} }{  \left | I \right|_{\sigma} }    \approx \left ( \frac{h}{\ell \left (I \right)} \right )^{\kappa+1} \frac{ \left | u_{\beta^*} \right | \left | K_{\beta^*} \right |_{\sigma} }{ \left |I \right|^{1 - \frac{\lambda}{n}} \left | I \right |_{\sigma } }  \]

Fixing $\delta$ and $2^{-m} = \frac{h}{\ell \left (I \right)}$ to be sufficiently small constants, we see that the first term dominates, that $T_{\sigma} \varphi (x)$ is of one consistent sign for $x \in J$ and 
\[
	\left | T_{\sigma} ^{\lambda} \varphi (x) \right | = \left | \int K ^{\lambda } (x,y) \varphi (y) d\sigma(y) \right | \approx \left ( \frac{h}{\ell \left (I \right)} \right )^{\kappa} \frac{ \left | u_{\beta^*} \right | \left | K _{\beta^*} \right | _{\sigma} }{ \left |I \right|^{1 - \frac{\lambda}{n}} \left | I \right| _{\sigma}}  \approx \left ( \frac{h}{\ell \left (I \right)} \right )^{\kappa} \frac{1}{\left |I \right|^{1 - \frac{\lambda}{n}}} \sum\limits_{\beta} \left | u_{\beta} \right | \frac{\left | K _{\beta} \right | _{\sigma}}{\left |I \right|_{\sigma}} \, . 
\]
 Because we have fixed $\delta, m$, which only depend on $\kappa$, $n$, the doubling constants for $\sigma$ and $\omega$ and the Calder\'on-Zygmund data for $T$, \emph{we now absorb any constants depending on $\delta, m$ into the symbols $\lesssim, \approx$ and $\gtrsim$,} i.e.,
 \[
	\left | T_{\sigma} ^{\lambda} \varphi (x) \right | \approx  \frac{1}{\left |I \right|^{1 - \frac{\lambda}{n}}} \sum\limits_{\beta} \left | u_{\beta} \right | \frac{\left | K _{\beta} \right | _{\sigma}}{\left | I \right |_{\sigma}} \, .
\]
By doubling of $\sigma$, we have $\left | K _{\beta} \right | _{\sigma} \approx \left | I \right |_{\sigma}$. Combined with the estimate $\sum\limits_{\beta} \left | u _{\beta} \right | \approx 1$, we in fact obtain the nondegeneracy condition \eqref{eq:nondegen_original}. This completes the proofs of Theorems \ref{main' Alpert} and \ref{main' p glob Alpert}.


\section{Concluding remarks}

For the orthonormal basis of Haar wavelets $\left\{
h_{Q}^{\mu}\right\}  _{Q\in\mathcal{D}}$ in $\mathbb{R}^{n}$ for
\emph{doubling} measures $\mu$, we obtain the best theorem possible for
admissible truncations $T^{\lambda}$ of \emph{gradient elliptic} fractional
Calder\'{o}n-Zygmund operators: for any dyadic
grid $\mathcal{D}$, the operator norm of $T^{\lambda} _{\sigma}$ from $L^{2}\left(  \sigma\right)
$ to $L^{2}\left(  \omega\right)  $ is comparable to the sum of the two Haar
testing characteristics.

Keeping with the assumption of doubling measures, little is known for
Calder\'{o}n-Zygmund operators that are not gradient elliptic. For the larger class of Stein elliptic Calder\'on-Zygmund operators, the norm inequality is characterized by two conditions, namely an $A_2$ condition and a cube testing condition \cite{AlSaUr}. As we showed in this paper, these latter two conditions are implied by the Haar testing condition when the measures are doubling. 


Keeping with the assumption of gradient elliptic operators, the situation for arbitrary locally finite positive Borel measures
$\mu$ in $\mathbb{R}^{n}$ is quite abysmal. In this general
case, we don't even know if the operator norm is comparable to the Haar testing characteristics
when $T^{\lambda}$ is an admissible truncation of the simplest singular integral, the
\emph{Hilbert transform} $H$. Or even if the operator norm of
$H$ is comparable to the sum of the two Haar testing characteristics, but
taken now over \emph{all} dyadic grids $\mathcal{D}$!

Finally, we mention that almost nothing is known about the more abstract problem of identifying those triples $\left(  T,\mathcal{F}%
_{1},\mathcal{F}_{2}\right)  $ of bounded operators $T$ (or collections of
such) from a Banach space $B_{1}$ to a Banach space $B_{2}$, and frames
$\mathcal{F}_{1}$ and $\mathcal{F}_{2}$ in these Banach spaces, respectively,
such that the operator norm of (each) $T$ is (uniformly) comparable to the sum
of the testing characteristics for $T$ associated to the frames
$\mathcal{F}_{1}$ and $\mathcal{F}_{2}$.

\begin{thebibliography}{999999999}

\bibitem[AlSaUr]{AlSaUr}\textsc{M. Alexis, E. Sawyer and I. Uriarte-Tuero,}%
\textit{\ A }$T1$\textit{\ theorem for general Calder\'{o}n-Zygmund operators
with doubling weights, and optimal cancellation conditions, II},
\texttt{arXiv:2111.06277}.

\bibitem[AlLuSaUr]{AlLuSaUr}\textsc{Michel Alexis, Jos\'{e} Luis Luna-Garcia,
Eric Sawyer, Ignacio Uriarte-Tuero }\textit{Stability of Weighted Norm
Inequalities, \texttt{arXiv:2208.08400v4}.}

\bibitem[AlLuSaUr3]{AlLuSaUr3}\textsc{Michel Alexis, Jos\'{e} Luis
Luna-Garcia, Eric Sawyer, Ignacio Uriarte-Tuero }\textit{The scalar} $T1$
\textit{theorem for pairs of doubling measures fails for Riesz transforms
when} $p>2$ \textit{is an even integer}, \texttt{arXiv:2308.15739}

\bibitem[CaHaLa]{CaHaLa}\textsc{Peter G. Cassaza, Deguang Han and David R.
Larsen, }\textit{Frames for Banach spaces, Contemporary Mathematics
\textbf{247} }(1999), 149-182.

\bibitem[Gro]{Gro}\textsc{Karlheinz Grochenig, }\textit{Describing functions:
Atomic decompositions versus frames, }Monatschefte fur Math. \textbf{112}
(1991) 1-41.

\bibitem[Hyt2]{Hyt2}\textsc{Hyt\"{o}nen, Tuomas, }\textit{The two weight
inequality for the Hilbert transform with general measures,
\texttt{arXiv:1312.0843v2}.}

\bibitem[HyVu]{HyVu}\textsc{Tuomas Hyt\"{o}nen and Emil Vuorinen, }\textit{A
two weight inequality between }$L^{p}\left(  \ell^{2}\right)  $\textit{\ and
}$L^{p}$, \texttt{arXiv:1608.07199v2}.

\bibitem[Lac]{Lac}\textsc{Michael T. Lacey, }\textit{\ Two weight inequality for the Hilbert transform: A real variable characterization, II}, Duke Math. J. Volume \textbf{163}, Number 15 (2014), 2821-2840.

\bibitem[LaSaShUr3]{LaSaShUr3}\textsc{Michael T. Lacey, Eric T. Sawyer,
Chun-Yen Shen, and Ignacio Uriarte-Tuero,} \textit{Two weight inequality for
the Hilbert transform: A real variable characterization I}, Duke Math. J,
Volume \textbf{163}, Number 15 (2014), 2795-2820.

\bibitem[LaSaUr2]{LaSaUr2}\textsc{Lacey, Michael T., Sawyer, Eric T.,
Uriarte-Tuero, Ignacio,} \textit{A Two Weight Inequality for the Hilbert
transform assuming an energy hypothesis, } Journal of Functional Analysis,
Volume \textbf{263} (2012), Issue 2, 305-363.

\bibitem[SaShUr7]{SaShUr7}\textsc{Sawyer, Eric T., Shen, Chun-Yen,
Uriarte-Tuero, Ignacio,} A \textit{two weight theorem for }$\alpha
$\textit{-fractional singular integrals with an energy side condition},
Revista Mat. Iberoam. \textbf{32} (2016), no. 1, 79-174.

\bibitem[SaShUr9]{SaShUr9}\textsc{Sawyer, Eric T., Shen, Chun-Yen,
Uriarte-Tuero, Ignacio,} A \textit{two weight fractional singular integral
theorem with side conditions, energy and }$k$\textit{-energy dispersed,}
Harmonic Analysis, Partial Differential Equations, Complex Analysis, Banach
Spaces, and Operator Theory (Volume 2) (Celebrating Cora Sadosky's life),
Springer 2017 (see also \texttt{arXiv:1603.04332v2}).

\bibitem[SaShUr10]{SaShUr10}\textsc{Sawyer, Eric T., Shen, Chun-Yen,
Uriarte-Tuero, Ignacio,} \textit{A good-}$\lambda$\textit{\ lemma, two weight
}$T1$\textit{\ theorems without weak boundedness, and a two weight accretive
global }$Tb$\textit{\ theorem,} Harmonic Analysis, Partial Differential
Equations and Applications (In Honor of Richard L. Wheeden), Birkh\"{a}user
2017 (see also \texttt{arXiv:1609.08125v2}).

\bibitem[NTV]{NTV}\textsc{F. Nazarov, S. Treil and A. Volberg,} \textit{The Bellman Functions and Two-Weight Inequalities for Haar Multipliers, }Journal of the American Mathematical Society , Oct., 1999, Vol. \textbf{12}, No. 4 (Oct., 1999), 909-928.

\bibitem[RaSaWi]{RaSaWi}\textsc{Robert Rahm, Eric T. Sawyer and Brett D.
Wick,} \textit{Weighted Alpert wavelets}, \texttt{arXiv:1808.01223v2}.

\bibitem[Saw6]{Saw6}\textsc{E. Sawyer}, \textit{A }$T1$\textit{\ theorem for
general Calder\'{o}n-Zygmund operators with comparable doubling weights and
optimal cancellation conditions}, Journal d'Analyse Math\'{e}matique
\textbf{146} no. 1 (2022), 205-297.

\bibitem[SaWi]{SaWi}\textsc{Eric T. Sawyer and Brett D. Wick,} \textit{Two
weight Sobolev norm inequalities for smooth Calderon-Zygmund operators and
doubling weights}, \texttt{arXiv:2207.14684}.

\bibitem[Ste2]{Ste2}\textsc{E. M. Stein,} \textit{Harmonic Analysis:
real-variable methods, orthogonality, and oscillatory integrals}%
,\textit{\ }Princeton University Press, Princeton, N. J., 1993.

%\bibitem[Vuo]{Vuo}\textsc{Emil Vuorinen, }Two weight $L^{p}$-inequalities\textit{\ for dyadic shifts and the dyadic square function}, Studia Math. \textbf{237} (1) \ (2017), 25-56.
\end{thebibliography}


\end{document}
